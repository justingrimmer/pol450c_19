\documentclass{beamer}

%\usepackage[table]{xcolor}
\mode<presentation> {
  \usetheme{Boadilla}
%  \usetheme{Pittsburgh}
%\usefonttheme[2]{sans}
\renewcommand{\familydefault}{cmss}
%\usepackage{lmodern}
%\usepackage[T1]{fontenc}
%\usepackage{palatino}
%\usepackage{cmbright}
  \setbeamercovered{transparent}
\useinnertheme{rectangles}
}
%\usepackage{normalem}{ulem}
%\usepackage{colortbl, textcomp}
\setbeamercolor{normal text}{fg=black}
\setbeamercolor{structure}{fg= black}
\definecolor{trial}{cmyk}{1,0,0, 0}
\definecolor{trial2}{cmyk}{0.00,0,1, 0}
\definecolor{darkgreen}{rgb}{0,.4, 0.1}
\newtheorem{prop}{Proposition}
\newtheorem{defn}{Definition}
\newtheorem{cor}{Corollary}
\newtheorem{lem}{Lemma}

\newcommand\indep{\protect\mathpalette{\protect\independenT}{\perp}}
\def\independenT#1#2{\mathrel{\rlap{$#1#2$}\mkern2mu{#1#2}}}

\usefonttheme{professionalfonts} % using non standard fonts for beamer
\usefonttheme{serif}

\beamertemplatesolidbackgroundcolor{white}  \setbeamercolor{alerted
text}{fg=red}
\usepackage{multirow}

\setbeamertemplate{caption}[numbered]\newcounter{mylastframe}

%\begin{beamerboxesrounded}[upper=uppercol,lower=lowercol,shadow=true]{Block}
%$A = B$.
%\end{beamerboxesrounded}}
\renewcommand{\familydefault}{cmss}
\usepackage[all]{xy}
\usepackage{tikz}
\usetikzlibrary{arrows}
%\usepackage{palatino}
%\usepackage{eulervm}
\usecolortheme{lily}
%\usepackage[latin1]{inputenc}

\title[Methodology III] % (optional, nur bei langen Titeln nötig)
{Political Methodology III: Model Based Inference}

\author{Justin Grimmer}
\institute[Stanford University]{Associate Professor\\Department of Political Science \\  Stanford University}
\vspace{0.3in}

\date{May 2nd, 2019}

\begin{document}
\begin{frame}
\titlepage
\end{frame}


\begin{frame}
\frametitle{Model Based Inference}

\begin{itemize}
\item[1)] Likelihood inference
\item[2)] Logit/Probit
\item[3)] Ordered Probit
\item[4)] Choice Models:
\item[5)] Count Models
\item[6)] Survival Models
\item[7)] Hypothesis Tests + Model Checking in Likelihood
\begin{itemize}
\item[-] Likelihood Ratios, Wald, and Score tests
\item[-] Model Checking: analysis of residuals, hat values, etc.
\end{itemize}
\end{itemize}

\end{frame}





\frame{

  \frametitle{What is Survival Analysis?}

  \begin{itemize}
  \item Analyze the length of time spent in a given state
  \item $Y_i \in [0, \infty)$: Duration, ``time to an event''
  \item Suppose $Y_{i}$ has density $f(y)$.
\pause
\medskip
  \item Example: Cabinet duration
    \begin{itemize}
      \item Are cabinets more likely to dissolve early or late?
      \item What factors predict the length of time until dissolution?
\pause
\smallskip
      \item King, Alt, Burns \& Laver (1990 AJPS) \dotfill Exponential model
      \item Warwick \& Easton (1992 AJPS) \dotfill Weibull model
      \item Warwick (1992 AJPS) \dotfill Cox PH model
      \item Diermeier \& Stevenson (1999 AJPS) \dotfill Competing risks model
    \end{itemize}
\pause
\medskip
  \item One of the most sophisticated subfields of statistical modeling, developed
    in multiple disciplines
  \item We will only be able to scratch the surface
 \end{itemize}

}

\frame{

  \frametitle{Survival Function}

  \begin{itemize}
  \item \alert{Survival function}: Probability of surviving at least up to time $y$
    $$S(y) \ \equiv \ \Pr(Y_i > y) \ = \ \int_{y}^{\infty} f(t)dt \ = \ 1 - \int_{0}^{y} f(t)dt \ = \  1-F(y)$$
  \item How likely am I to live at least $y$ years?
\pause
\medskip
  \item Properties:
    \begin{itemize}
      \addtolength{\itemsep}{0.25\baselineskip}
    \item $S(0) = 1$ and $S(\infty) = 0$; monotonically decreasing
\pause
\smallskip
    \item Area under $S(y)$ is the average survival time:
      \begin{eqnarray}
      \text{E}(Y_i) & = &  \int_{0}^{\infty} y f(y) dy  \nonumber \\
      & = & y \left(F(y)|^{\infty}_{0}\right)  - \int_{0}^{\infty} F(y) dy \nonumber \\
      & = & \int_{0}^{\infty} (1- F(y)) dy \nonumber \\
      & = & \int_{0}^{\infty} S(y)dy \nonumber
      \end{eqnarray}
\pause
\vspace{-.75\baselineskip}
  \end{itemize}
  \end{itemize}
}

\begin{frame}
\frametitle{Survival Function}

\begin{itemize}
   \item One-to-one relationships with density and probability:
    \begin{eqnarray*}
     & & f(y) \ = \ -\frac{d}{dy}S(y) \quad \text{and} \quad S(y) \ = \ \int_y^\infty f(t) dt \\
     & & \Pr(y \le Y_i < y + h) \ = \ S(y)-S(y+h)
    \end{eqnarray*}
\end{itemize}

\end{frame}


\frame{
\frametitle{Survival Function}
\begin{center}
\only<1|handout:0>{\includegraphics[height=.8\textheight]{hazardsurvival-1.pdf}}%
\only<2|handout:0>{\includegraphics[height=.8\textheight]{hazardsurvival-2.pdf}}%
\only<3|handout:1>{\includegraphics[height=.8\textheight]{hazardsurvival-3.pdf}}
\end{center}
}


\frame{

  \frametitle{Hazard Function}

  \begin{itemize}
  \item \alert{Hazard function}: Instantaneous rate of leaving a state at time
    $t$ conditional on survival up to that time
    $$ \lambda(y) \ \equiv \ \lim_{h \downarrow 0} \frac{\Pr(y \le Y_i < y + h \mid Y_i \ge y)}{h}
    \ = \ \frac{f(y)}{S(y)} $$
  \item ``Force of mortality'' --- what is the `risk' that I die at time $y$ given that I have lived up until $y$?
  \item Difficult to directly interpret, but useful for model checking, etc.
\medskip
\pause
    \item One-to-one relationship with survival function:
      $$ \lambda(y) \ = \ - \frac{d}{dy} \log S(y)
      \quad \text{and} \quad S(y) \ = \ \exp\left(-\int_0^y \lambda(t) dt \right)  $$
  % \item \alert{Cumulative hazard function}:
  %   \begin{eqnarray*}
  %     \Lambda(y) & \equiv & \int_0^y \lambda(t) dt \ = \ -\log S(y)
  %   \end{eqnarray*}
  \end{itemize}

}

\frame{
\frametitle{Hazard Function}
\begin{center}
\only<1|handout:0>{\includegraphics[height=.8\textheight]{hazardsurvival-3.pdf}}%
\only<2|handout:1>{\includegraphics[height=.8\textheight]{hazardsurvival-4.pdf}}
\end{center}
}


\frame{

  \frametitle{Quantities of Interest}

  \begin{itemize}
  \addtolength{\itemsep}{0.25\baselineskip}
  \item Shape of the survival curve
\pause
\medskip
  \item \alert{Expected} (remaining) \alert{time to event} ($=$ life expectancy at age $y$):
   $$ \mu(y) \ \equiv \ \text{E}(Y_i - y \mid Y_i > y)
   \ = \ \frac{1}{S(y)} \int_y^\infty S(t) dt $$
  \item Given that I'm alive at $y$, how much longer should I expect to live?
\medskip
\pause
\item Predicted differences in the above
\medskip
\pause
\item Causal effects on survival outcomes:
  \begin{itemize}
  \addtolength{\itemsep}{.25\baselineskip}
  \item \alert{One-shot treatment} administered at the beginning of study period \\
   --- needs conditional ignorability given observed pre-trial covariates
  \item \alert{Time-varying treatment}, possibly given in response to covariates \\
  --- needs ``sequential ignorability''
  \end{itemize}
  \end{itemize}

}

\frame{

  \frametitle{Censoring}

  \begin{itemize}

  \item Observation is \alert{right-censored} when only the lower bound of
    duration is known: $Y_i \in (c, \infty)$
\pause
\medskip
  \item The \alert{independent censoring} assumption: Censored observations
    do not systematically differ from complete observations in terms of hazard rates
  \item A sufficient condition: $Y_i \indep C_i \mid X_i$ where $C_i =$ time to censoring
  \item Either $Y_i$ or $C_i$ is actually observed

\pause
  \item Examples:
    \begin{itemize}
      \item Random attrition of study sample
      \item Study begins and ends at exogenously fixed calendar dates
      \item Study ends after fixed duration (type I censoring)
      \item Study ends after a fixed number of failures (type II censoring)
    \end{itemize}

\pause
\medskip
\item Other types of censoring (left, interval) are less common

 \end{itemize}

}


% \frame{
% \frametitle{Typical Observation Schemes for Duration Analysis}
% \begin{center}
% \includegraphics[width=.8\textwidth]{censoring-1.pdf}
% \end{center}
% Q. Which observations are right/left/interval censored? \pause \\
% A. 1 is not censored; 2--4 are {\it all censored}.

% }

% %\frame{
% %\frametitle{Typical Observation Schemes for Duration Analysis}
% %\begin{center}
% %\includegraphics[width=.8\textwidth]{pics/censoring-3.pdf}
% %\end{center}
% %Q. Which observations are right/left/interval censored? \pause \\
% %A. 1-2 are not censored; 3 is right censored; 4 is left censored.
% %
% %}

% \frame{
% \frametitle{Typical Observation Schemes for Duration Analysis}
% \begin{center}
% \includegraphics[width=.8\textwidth]{censoring-2.pdf}
% \end{center}
% Q. Which observations are right/left/interval censored? \pause \\
% A. 1 is interval censored; 2--4 are right censored.

% }


\frame{

\frametitle{Cabinet Duration Example: Censoring}
King, Alt, Burns, and Laver (1990 AJPS):
\begin{itemize}
\item $Y_i$: Duration of parliamentary cabinets, $n=314$ \pause
\begin{center}
\includegraphics[scale=0.5]{cabinet.pdf}
\end{center}
\item Notice the ``bump''?
\medskip \pause
\item Some cabinets end their lives ``naturally''
\item Others end because of constitutional interelection periods (CIEP)
\medskip \pause
\item King et al. treat CIEPs as censored observations
\item But is this ``censoring'' independent?
\end{itemize}
}


\section{Nonparametric Estimation of Survival Function}

\frame{

  \frametitle{Discrete Time Approximation}

  \begin{itemize}
  \item Time is continuous but we observe discrete time: $t_1 < t_2 < \cdots$
\pause
  \item Density function: $f(t_j) = \Pr(Y_i = t_j)$
  \item Survival function: $S(t_j) = \Pr(Y_i \alert{>} t_j) = \sum_{\{k:\ t_k > t_j\}} f(t_k)$
\pause
  \item Hazard function: $\lambda(t_j) = \Pr(Y_i = t_j \mid Y_i \alert{\ge} t_j)  =  f(t_j)\big/ S(t_{\alert{j-1}})$
  % \item Cumulative hazard function: $\Lambda(y)=\sum_{\{j:\ t_j \le y\}} \lambda(t_j)$
\pause
\medskip
  \item Key relationships:
   \begin{itemize}
    \addtolength{\itemsep}{0.5\baselineskip}
    \item $S(t_j) = \prod_{k=1}^j (1-\lambda(t_k))$
\pause
    \item $f(t_j) = S(t_{j-1}) - S(t_j) = \lambda(t_j)\prod_{k=1}^{j-1}(1-\lambda(t_k))$
    \end{itemize}
\medskip
\pause
\item Expected remaining time to event:
  \begin{eqnarray*}
  \mu(t_j) & = & \text{E}(Y_i - t_j\mid Y_i > t_j) \\
  & = & \frac{1}{S(t_j)}\sum_{k=j+1}^\infty(t_k-t_j)f(t_k)
  \ = \ \frac{1}{S(t_{j})}\sum_{k=j}^\infty (t_{k+1} - t_{k})S(t_{k})
  \end{eqnarray*}
\end{itemize}


}

\frame{

  \frametitle{Estimating the Survival Curve Without a Model}

  \begin{itemize}
  \item Goal: Get the sense of what $S(t_j)$ looks like before introducing $X_i$
\pause
\medskip
  \item Easy if no censoring; just count \# of units failing at each $t_j$
  \item Censored observations make things a bit more complicated
\pause
\medskip
  \item Setup:
    \begin{itemize}
      \addtolength{\itemsep}{0.25\baselineskip}
      \item Observed failure times: $t_1 < t_2 < \cdots < t_J$
      \item $d_j = $ \# of units that failed at time $t_j$
      \item $m_j = $ \# of units censored at time $t_j$
      \item $r_j = \sum_{k=j}^J (d_k + m_k)$ \\
        \quad $ = $ \# of units \alert{at risk} at time $t_j$, i.e.,
        those that have neither failed nor been censored
        until right before $t_j$
   \end{itemize}
  \medskip
\pause
  \item A natural estimate for the hazard function will then be:
    $$ \hat\lambda(t_j) \ = \ \widehat{\Pr}(Y_i = t_j \mid Y_i \geq t_j) \ = \ \frac{d_j}{r_j} $$
  \item In fact, this is the MLE of $\lambda(t_j)$
  \end{itemize}


}

\frame{

  \frametitle{Kaplan-Meier Estimator}

  \begin{itemize}
  \item This leads to the \alert{Kaplan-Meier estimator}:
$$\widehat{S}(t_j) \ = \ \prod_{k=j}^J(1-\hat\lambda(t_k)) \ = \ \prod_{k=j}^J \frac{r_k-d_k}{r_k}$$

\pause
\bigskip

  \item Using the MLE derivation for $\hat\lambda(t_j)$, we obtain the Hessian-based
    estimate of the asymptotic variance:

    $$\widehat{\text{Var}(\widehat{S}(t_j))} \ = \ \widehat{S}^2(t_j)\sum_{k=j}^J\frac{d_k}{r_k(r_k-d_k)}$$

  \end{itemize}

}

\frame{

  \frametitle{Example: Time Until Commitment to IMF Article VIII}

  \begin{center}
    \includegraphics[scale=0.6]{Kaplan-Meier.pdf}
  \end{center}
  Simmons (2000 APSR)
}

\section{Parametric Regression Models}

\frame{

\frametitle{Exponential Regression Model}

\begin{itemize}
\item Suppose that failures occur according to a Poisson process \\
(i.e. continuously, independently, and with constant probability)
\item Then the ``time to an event'' follows the \alert{exponential} distribution
\medskip
\pause
\item Model: $Y_i\mid X_i \sim_{\text{ind}} \text{Exponential}(\mu_i)$ where $\mu_i = \exp(X_i^{'} \beta)$

\item Density: $f(y \mid \mu_i) = \frac{1}{\mu_i} \exp(-y/\mu_i)$

\item Mean $\text{E}(Y_i \mid \mu_i) =\mu_i$ and Variance $\text{Var}(Y_i \mid \mu_i) = \mu_i^2$

\item Survival function: $S(y) = \exp(-y/\mu_i)$

\item Hazard function: $\lambda(y) = 1/\mu_i = \exp(-X_i^{'}\beta)$ \ (constant in $y$)

\pause
\medskip

\item A common alternative parameterization: $\gamma_i \equiv 1/\mu_i$
\item This changes nothing except that the sign of $\beta$ gets reversed

\pause
\medskip

\item The \alert{constant hazard} assumption: $\lambda_i$ does not vary across time
\end{itemize}

}

\frame{

\frametitle{MLE for the Exponential Model with Censoring}

\begin{itemize}
\item Censoring indicator: $D_i = 1$ if censored
\item $Y_i$ is the censoring time (rather than failure time) if $D_i = 1$
\medskip
\pause
\item Likelihood function:
  \begin{eqnarray*}
    L_n(\beta \mid Y, X, D)
    & = & \prod_{i=1}^n \underbrace{\left\{f(Y_i\mid \mu_i)\right\}^{1-D_i}}_{\text{uncensored}} \cdot
        \underbrace{\left\{S(Y_i\mid \mu_i)\right\}^{D_i}}_{\text{censored}} \\
    & = & \prod_{i=1}^n \left\{\frac{1}{\mu_i}\exp(-Y_i/\mu_i)\right\}^{1-D_i}
        \left\{\exp(-Y_i/\mu_i)\right\}^{D_i} \\
    & = & \prod_{i=1}^n \exp\left\{-(1-D_i)X_i^\top\beta\right\} \exp\left\{-\exp(-X_i^{'}\beta)Y_i\right\}
  \end{eqnarray*}
\pause
\item Log-likelihood, score and Hessian can be calculated as usual
\end{itemize}

}


\frame{

\frametitle{Weibull Regression Model}

\begin{itemize}
\item The constant hazard assumption is often too restrictive
\item The \alert{Weibull} model relaxes the assumption by introducing a ``shape'' parameter

\pause
\medskip

\item Model: $Y_i\mid X_i \sim_{\text{ind}} \text{Weibull}(\mu_i, \alpha)$ where $\mu_i = \exp(X_i^{'}\beta)$ and $\alpha > 0$
\item Density: $f(y\mid \mu_i,\alpha) = \frac{\alpha}{\mu_i^\alpha}y^{\alpha-1}\exp\{-(y/\mu_i)^\alpha\}$
\item Reduces to the exponential model when $\alpha = 1$
\item Survival function: $S(y) = \exp\{-(y/\mu_i)^\alpha\}$
\item Hazard function: $\lambda(y) \ = \ \frac{\alpha}{\mu_i^\alpha}y^{\alpha-1}$

\pause
\medskip

\item The \alert{monotonic hazard} assumption:
  increasing (decreasing) if $\alpha > 1$ (if $\alpha < 1$)
\medskip
\item Other parametric regression models:
    \begin{itemize}
    \item Gompertz: Monotonic hazard
    \item Log-normal, Log-logistic: Inverse U-shaped hazard
    \end{itemize}
\end{itemize}

}

\frame{
 \frametitle{Cabinet Duration Example: Exponential or Weibull?}
 King et al. (Exponential) vs. Warwick and Easton (Weibull)
 \begin{itemize}
 \item Comparing density functions:
 \end{itemize}
 \vspace{-\baselineskip}
 \begin{center}
  \includegraphics[width=.85\textwidth]{cabinet-density.pdf}
\end{center}

}

\frame{
 \frametitle{Cabinet Duration Example: Exponential or Weibull?}
 King et al. (Exponential) vs. Warwick and Easton (Weibull)
 \begin{itemize}
 \item Comparing hazard functions:
 \end{itemize}
 \vspace{-\baselineskip}
 \begin{center}
 \includegraphics[width=.85\textwidth]{cabinet-hazard.pdf}
 \end{center}
}

\section{Cox Proportional-Hazard Model}

\frame{

  \frametitle{Semi-Parametric Regression for Survival Data}

  \begin{itemize}
  \item Less restriction on the hazard function
  \item Time-varying covariates to further model stochastic risks

\pause
\medskip
  \item Note that both exponential and Weibull models are \alert{proportional hazard models}:
\vspace{-.08in}
  $$ \lambda(y\mid X_i) \ = \ \underbrace{\lambda_0(y)}_{\text{baseline\ hazard}} \exp(X_i^{'}\beta^\ast) $$
  where $\lambda_0(y) \ = \ {\scriptsize \begin{cases} 1 & \text{(exponential)} \\ \alpha y^{\alpha-1} & \text{(Weibull)} \end{cases}}$ \ and \ $\beta^\ast = -\alpha\beta$

\pause
\bigskip
  \item The \alert{Cox Proportional Hazard Model} generalizes this model:

    $$\lambda(y\mid X_{i}(y)) \ = \ \lambda_0(y) \exp(X_i(y)^{'} \beta^\ast)$$
    where
    \begin{itemize}
    \item $\lambda_0(y)$: Nonparametric baseline hazard common to all $i$ across $t$
    \item $X_i(y)$: (Potentially) time-varying covariates
    \end{itemize}
    {\footnotesize (Note: We omit $\ast$ from hereon)}

  \end{itemize}

}


\begin{frame}
\frametitle{Example: Hazards Accommodated by the Cox Model}
\begin{center}
\includegraphics[width=.3\textwidth]{CoxHazard1.pdf}%
\includegraphics[width=.3\textwidth]{CoxHazard2.pdf}%
\includegraphics[width=.3\textwidth]{CoxHazard3.pdf}
\end{center}
\begin{itemize}
\item The Cox PH model allows flexible shapes of hazard functions
\medskip
\item Suppose we have one binary predictor $x \in \{0,1\}$ to model $y$:
\smallskip
\begin{enumerate}
\addtolength{\itemsep}{.25\baselineskip}
\item $\lambda(y\mid x) \ = \ \lambda_0(y) \exp(x\beta)$ \hfill ------ no time-varying covariate
\item $\lambda(y\mid x) \ = \ \lambda_0(y) \exp\left[x\beta_1 + xy\beta_2\right]$ \hfill ------ interaction with time trend
\item $\lambda(y\mid x) \ = \ \lambda_0(y) \exp\left[x \beta_1 + x(1.5-y)\boldsymbol{1}\{y\leq 1.5\}\beta_2\right]$ \\
      \flushright ------ allowing high initial risk
\end{enumerate}
\medskip
{\footnotesize
\item[] Note: In the figures, the \alert{relative risk} (RR) stands for:
$$ RR \ = \  \frac{\lambda(y\mid x=1)}{\lambda(y\mid x=0)}
      \ = \ \exp\left[g(y\mid x=1) - g(y\mid x=0)\right]$$
}
\end{itemize}
\end{frame}


\begin{frame}
\frametitle{Estimation of Cox Proportional Hazard Models}
\small
\vspace{-.5\baselineskip}
\begin{itemize}
  \item Joint MLE for $\lambda_0(y)$ and $\beta$ is difficult (because $\lambda_0(y)$ is nonparametric)
  \item Instead, consider the \alert{partial likelihood function} which only contains
    information about non-censored observations
\pause
  \item Because of the independent censoring assumption, this should give us
    a consistent (although not efficient) estimate for $\beta$
\pause
\smallskip
  \item For now, suppose that no two observations fail at the same time
  \item This implies we can unambiguously index observations by $j$
\pause
\medskip
  \item Under this assumption, the partial likelihood function turns out to be:
  {\footnotesize
    \begin{eqnarray*}
   L_P(\beta) & = & \prod_{j=1}^J \frac{\exp(X_i(t_j)^\top\beta)}
             {\sum_{k\in R(t_j)}\exp(X_{k}(t_j)^\top\beta)}
    \end{eqnarray*}
    where $R(t_j) = \text{risk set at time } t_j$
    }
\pause
%\vspace{-.5\baselineskip}
  \item Note that $\lambda_0(y)$ drops out of the partial likelihood
\pause
\smallskip
  \item Take the log and maximize to obtain the estimate of $\beta$
\end{itemize}
\end{frame}







\begin{frame}

\huge

Hypothesis Testing

\end{frame}



\begin{frame}
\frametitle{Simple Example: Antiobama Speech}

We'll  use the speech data from the problem set, as follows:

\begin{itemize}
\item[-] $Y_{i} = 1 $ if representative says {\tt obamacare} or {\tt big government} during the year, 0 otherwise
\item[-] $\boldsymbol{X}_{i} = (1, I(\text{Year} = 2010)_{i}, \text{Democrat}_{i}, \text{DW-Nom}_{i}) $
\end{itemize}

\begin{eqnarray}
Y_{i}  &  \sim &   \text{Bernoulli}(\pi_{i}) \nonumber \\
\pi_{i}     & = & \text{logit}^{-1}(\boldsymbol{X}_{i}^{'} \boldsymbol{\beta})  = \frac{1}{1 + \exp(-\boldsymbol{X}_{i}^{'} \boldsymbol{\beta}) } \nonumber
\end{eqnarray}


Which covariates do we include? $\leadsto$ depends on goal.  \\
\begin{itemize}
\item[-] Predictive goal $\leadsto$ replicate task
\item[-] Model fitting $\leadsto$ do covariates increase likelihood? Can we drop them?
\end{itemize}

\end{frame}




\subsection{Hypothesis Testing}

\begin{frame}
\frametitle{Hypothesis Testing --- Likelihood Ratio Test}
\begin{itemize}[<+->]
\item Null ($H_0$): $h_1(\boldsymbol{\beta}) = \cdots = h_Q(\boldsymbol{\beta}) = 0$ ($Q$ equality constraints)
\item Alternative ($H_1$): No such constraints
\medskip
\item Let $\widehat{\boldsymbol{\beta}}_{R} = \widehat{\boldsymbol{\beta}}_{MLE\mid H_0}$ (restricted MLE) and
$\widehat{\boldsymbol{\beta}}_{UR} = \widehat{\boldsymbol{\beta}}_{MLE}$ (original MLE)
\item \alert{Likelihood ratio} (LR) test: If $H_0$ is true, $L(\widehat{\boldsymbol{\beta}}_R)$
should be equal to $L(\widehat{\boldsymbol{\beta}}_{UR})$ except for sampling variability
\item LR statistic:
    $$ LR(Y) \ \equiv \ -2\log\frac{L(\widehat{\boldsymbol{\beta}}_{R})}{L(\widehat{\boldsymbol{\beta}}_{UR})}
     \ = \ 2\left[\ell(\widehat{\boldsymbol{\beta}}_{UR}) - \ell(\widehat{\boldsymbol{\beta}}_{R})\right] $$
\item We can show that $LR(Y) \stackrel{d}{\longrightarrow} \chi_Q^2$
\item Works for testing any nested models \\
    --- model under $H_0$ has to be a special case of model under $H_1$
\end{itemize}

\end{frame}

\begin{frame}

\small
\begin{semiverbatim}

un\_rest\_reg<- glm(once$\sim$two\_10 + dem + dw\_nom,

\hspace{0.15in} data = speech\_dat, family = binomial(link = logit))


rest\_reg<- glm(once$\sim$1, family= binomial(link = logit))


\#\#calculating the likelihood ratio

log\_lik<- function(pars, X, Y)\{

  \hspace{0.15in} y.tilde<- X\%*\%pars

  \hspace{0.15in} probs<- plogis(y.tilde)

  \hspace{0.15in} log\_out<- Y\%*\%log(probs) + (1-Y)\%*\%log(1 - probs)

  \hspace{0.15in} return(log\_out)

  \}

 X<- cbind(1, two\_10, dem, speech\_dat\$dw\_nom)


 un\_rest<- log\_lik(un\_rest\_reg\$coef, X, once)


 rest<- log\_lik(rest\_reg\$coef, as.matrix(rep(1, nrow(X))), once)


 > 2 * un\_rest - 2*rest

\hspace{0.15in}        [,1]

[1,] 433.996








 \#\#get the same statistic automatically from glm

 diff<- un\_rest\_reg\$null.deviance - un\_rest\_reg\$deviance

 1 - pchisq(diff, 3) \#\#very small!

[1] 0

\end{semiverbatim}



\end{frame}

\begin{frame}

\begin{semiverbatim}
 > 2 * un\_rest - 2*rest

\hspace{0.15in}        [,1]

[1,] 433.996

 \#\#get the same statistic automatically from glm

 diff<- un\_rest\_reg\$null.deviance - un\_rest\_reg\$deviance

 1 - pchisq(diff, 3) \#\#very small!

[1] 0

\end{semiverbatim}


\end{frame}


\begin{frame}
\frametitle{Hypothesis Testing --- Wald Test}
\begin{itemize}[<+->]
\item \alert{Wald test}: If true, the null $h_1(\boldsymbol{\beta})=\cdots =h_Q(\boldsymbol{\beta})=0$
should approximately hold even if we substitute $\widehat{\boldsymbol{\beta}}_{UR}$ for $\boldsymbol{\beta}$
\item Wald statistic: Use asymptotic distribution of $\widehat{\boldsymbol{\beta}}$ and representation of restrictions, properties of normal distribution to obtain form
{\small
    $$ W \ \equiv \ h(\widehat{\boldsymbol{\beta}}_{UR})^{'} \left[\left(\frac{\partial h(\boldsymbol{\beta})}
    {\partial \boldsymbol{\beta}}\Bigl|_{\boldsymbol{\beta}=\widehat{\boldsymbol{\beta}}_{UR}} \right)^{'}
    \widehat{\text{Var}(\widehat{\boldsymbol{\beta}}_{UR})}
    \left(\frac{\partial h(\boldsymbol{\beta})}{\partial\boldsymbol{\beta}}\Bigl|_{\boldsymbol{\beta}=\widehat{\boldsymbol{\beta}}_{UR}}\right)\right]^{-1}
    h(\widehat{\boldsymbol{\beta}}_{UR})$$
}

\item The ``meat'' $\simeq \widehat{\text{Var}(h(\widehat{\boldsymbol{\beta}}_{UR}))}$
(\alert{Delta method} )
\item Choose any $\widehat{\text{Var}(\widehat{\boldsymbol{\beta}}_{UR})}$ as appropriate (e.g. Huber-White)
\item We can show that $W \stackrel{d}{\longrightarrow} \chi_Q^2$
\bigskip
\item An important special case: $Q=1$ and $H_0: \beta = 0$
\item In this case, we can use the \alert{$z$ statistic}:
{\small
$$ z \ = \ W^{1/2} \ = \ \frac{\widehat{\boldsymbol{\beta}}_{UR}}{\text{s.e.}(\widehat{\boldsymbol{\beta}}_{UR})}
 \ \stackrel{d}{\longrightarrow} \ \text{N}(0,1)$$
}
%where $s.e.(\widehat{\beta}_{UR})$ is the sqrt of the diagonal element of $\widehat{\text{Var}(\widehat{\beta}_{UR})}$
\end{itemize}

\end{frame}


\begin{frame}

\small
\begin{semiverbatim}

> un\_rest\_reg\$coef\%*\%solve(vcov(un\_rest\_reg))\%*\%un\_rest\_reg\$coef

         [,1]

[1,] 225.2437


> 1 - pchisq(225.2437, 3)

[1] 0


\end{semiverbatim}



\end{frame}



\begin{frame}
\frametitle{Hypothesis Testing --- Score Test}
\begin{itemize}[<+->]
\item At the unrestricted MLE $\hat\theta_{UR}$, $\sum_{i=1}^{N} s_{i} (\widehat{\boldsymbol{\beta}}_{UR})=s(\widehat{\boldsymbol{\beta}}) = 0$ by construction
\item \alert{Score test}: If the null is true, $s(\widehat{\boldsymbol{\beta}}_{R})$
  should also equal zero except for sampling variability
\item Score statistic: Use asymptotic distribution and properties of normal distribution to  ``standardize" $s(\widehat{\boldsymbol{\beta}}_{R})$
$$ LM \ = \ s(\widehat{\boldsymbol{\beta}}_{R})^{'} \widehat{\text{Var}(\widehat{\boldsymbol{\beta}}_{R})} s(\widehat{\boldsymbol{\beta}}_{R})
\ \stackrel{d}{\longrightarrow} \ \chi_Q^2$$

\item For $\widehat{\boldsymbol{\beta}}_{QMLE}$, the expression is more complicated
\item Also known as the \alert{Lagrange multiplier} (LM) test due to an alternative derivation

\end{itemize}

\end{frame}



\begin{frame}

\small
\begin{semiverbatim}

 score\_func<- function(coef, X, Y)\{

  \hspace{0.15in} y.tilde<- X\%*\%coef

  \hspace{0.15in} probs<- plogis(y.tilde)

 \hspace{0.15in}  out<- t(Y - probs)\%*\%X

 \hspace{0.15in}  return(out)
   \}



> round(score\_func(un\_rest\_reg\$coef, X, once), 2)


[1,] 0      0   0 0


rest\_score<- score\_func(c(rest\_reg\$coef, 0, 0, 0), X, once)


> round(rest\_score,2)

[1,] 0 -6.30 -128.92 129.51

\end{semiverbatim}
\end{frame}


\begin{frame}
\small
\begin{semiverbatim}

 hess\_func<- function(coef, X, Y)\{

   \hspace{0.15in}  y.tilde<- X\%*\%coef

  \hspace{0.15in}   probs<- plogis(y.tilde)

  \hspace{0.15in}   base<- matrix(0, nrow = len(coef), ncol = len(coef))

   \hspace{0.15in}  for(z in 1:nrow(X))\{

   \hspace{0.25in}    base<- base + probs[z]*(1 - probs[z])* X[z,]\%*\%t(X[z,])

     \hspace{0.15in}  \}

     \hspace{0.15in} return(base)

   \}


 rest\_hess<- solve(hess\_func(c(rest\_reg\$coef, 0, 0, 0), X, once))

>rest\_score\%*\%rest\_hess\%*\%t(rest\_score)

[1,] 395.0382

> 1- pchisq(395, 3)

[1] 0

\end{semiverbatim}

\end{frame}


\begin{frame}
\frametitle{Comparing The Three Tests}
\vspace{-.2in}
\begin{center}
\includegraphics[width=\textwidth]{threetests.jpg}
\end{center}
\vspace{-.2in}
\begin{itemize}
\item All asymptotically equivalent
\item But can be quite different in small samples
\end{itemize}
\pause
{\small
\begin{tabular}{cll}
\hline\hline
   & {\it Pros}                           & {\it Cons} \\
\hline\hline
LR & Most powerful (Neyman-Pearson) & Must compute both $\hat\theta_{UR}$ and $\hat\theta_{R}$ \\
   &                                & Cannot be easily robustified \pause \\
\hline
W  & Only need $\hat\theta_{UR}$     & Not invariant to transformation \\
   & Easily robustified by sandwich & (e.g. $\theta_1/\theta_2 = 1$ vs. $\theta_1=\theta_2$) \pause \\
\hline
LM & Only need $\hat\theta_{R}$      & $\hat\theta_{R}$ often difficult to estimate \\
\hline\hline
\end{tabular}
}
\end{frame}


\begin{frame}

\huge

\begin{itemize}
\item Model diagnostics
\item AIC/BIC
\item Cross Validation
\item Midterm
\end{itemize}

\end{frame}

\end{document}


\begin{frame}
\frametitle{Example}








\end{frame}



\begin{frame}
\frametitle{Likelihood Ratio Test}



\end{frame}



\begin{frame}
\frametitle{Wald Test}



\end{frame}



\begin{frame}



\end{frame}


\begin{frame}


\end{frame}




\end{document}


\item Suppose we have $Q$ restrictions for $p$ covariates.  Include this in matrix $\boldsymbol{R}$.  Under $H_{0}$ we have:

\begin{eqnarray}
H_{0} & : &  \boldsymbol{R} \boldsymbol{\beta}  = r \nonumber \\
H_{1} & : & \boldsymbol{R} \boldsymbol{\beta} \neq r \nonumber
\end{eqnarray}

\item Under the null:
\begin{eqnarray}
\sqrt{n}(\widehat{\boldsymbol{\beta}}_{UR} - \boldsymbol{\beta}) & \rightarrow^{D} & \text{Normal}\left(0, \widehat{\text{Var}(\widehat{\boldsymbol{\beta}}_{UR})}\right) \nonumber \\
R \sqrt{n}(\widehat{\boldsymbol{\beta}}_{UR} - \boldsymbol{\beta}) =  & \rightarrow^{D} & \text{Normal}(0, \boldsymbol{R}\widehat{\text{Var}(\widehat{\boldsymbol{\beta}}_{UR})} \boldsymbol{R}^{'})\nonumber
\end{eqnarray}
\end{itemize}

\end{frame}
\end{document}
