\documentclass{beamer}

%\usepackage[table]{xcolor}
\mode<presentation> {
  \usetheme{Boadilla}
%  \usetheme{Pittsburgh}
%\usefonttheme[2]{sans}
\renewcommand{\familydefault}{cmss}
%\usepackage{lmodern}
%\usepackage[T1]{fontenc}
%\usepackage{palatino}
%\usepackage{cmbright}
  \setbeamercovered{transparent}
\useinnertheme{rectangles}
}
%\usepackage{normalem}{ulem}
%\usepackage{colortbl, textcomp}
\setbeamercolor{normal text}{fg=black}
\setbeamercolor{structure}{fg= black}
\definecolor{trial}{cmyk}{1,0,0, 0}
\definecolor{trial2}{cmyk}{0.00,0,1, 0}
\definecolor{darkgreen}{rgb}{0,.4, 0.1}
\newtheorem{com}{Comment}
\newtheorem{lem} {Lemma}
\newtheorem{prop}{Proposition}
\newtheorem{thm}{Theorem}
\newtheorem{defn}{Definition}
\newtheorem{cor}{Corollary}
\newtheorem{obs}{Observation}
 \numberwithin{equation}{section}


\newcommand\indep{\protect\mathpalette{\protect\independenT}{\perp}}
\def\independenT#1#2{\mathrel{\rlap{$#1#2$}\mkern2mu{#1#2}}}

\usefonttheme{professionalfonts} % using non standard fonts for beamer
\usefonttheme{serif}

\beamertemplatesolidbackgroundcolor{white}  \setbeamercolor{alerted
text}{fg=red}
\usepackage{multirow}

\setbeamertemplate{caption}[numbered]\newcounter{mylastframe}

%\begin{beamerboxesrounded}[upper=uppercol,lower=lowercol,shadow=true]{Block}
%$A = B$.
%\end{beamerboxesrounded}}
\renewcommand{\familydefault}{cmss}
\usepackage[all]{xy}
\usepackage{tikz}
\usetikzlibrary{arrows}
%\usepackage{palatino}
%\usepackage{eulervm}
\usecolortheme{lily}
%\usepackage[latin1]{inputenc}

\title[Methodology III] % (optional, nur bei langen Titeln nötig)
{Political Methodology III: Model Based Inference}

\author{Justin Grimmer}
\institute[Stanford University]{Professor\\Department of Political Science \\  Stanford University}
\vspace{0.3in}

\date{May 21st, 2019}

\begin{document}
\begin{frame}
\titlepage
\end{frame}


\begin{frame}
\frametitle{Model Based Inference}

\begin{itemize}
\item[1)] Likelihood inference
\item[2)] Machine Learning
\begin{itemize}
\item[a)] Unsupervised Latent Features
\item[b)] Classification/Prediction/Regression
\end{itemize}
\end{itemize}

\end{frame}

\begin{frame}
\frametitle{Regression models}

Suppose we have $N$ documents, with each document $i$ having label $y_{i} \in \{-1, 1\}\leadsto\{$not, credit claiming$\}$ \pause \\
\invisible<1>{We represent each document $i$ is $\boldsymbol{x}_{i} = (x_{i1}, x_{i2}, \hdots, x_{iJ})$. } \pause  \\

\begin{eqnarray}
\invisible<1-2>{f(\boldsymbol{\beta}, \boldsymbol{X}, \boldsymbol{Y})  & = & \sum_{i=1}^{N}\left( y_{i} - \boldsymbol{\beta}^{'} \boldsymbol{x}_{i} \right)^{2}  \nonumber \\} \pause
\invisible<1-3>{\widehat{\boldsymbol{\beta} } & = & \text{arg min}_{\boldsymbol{\beta}} \left\{\sum_{i=1}^{N}\left( y_{i} - \boldsymbol{\beta}^{'} \boldsymbol{x}_{i} \right)^{2}\right\} \nonumber \\} \pause
 \invisible<1-4>{& = & \left( \boldsymbol{X}^{'}\boldsymbol{X}   \right)^{-1}\boldsymbol{X}^{'}\boldsymbol{Y} \nonumber } \pause
\end{eqnarray}

\invisible<1-5>{Problem: \\} \pause
\begin{itemize}
\invisible<1-6>{\item[-] $J$ will likely be large (perhaps $J> N$)} \pause
\invisible<1-7>{\item[-] There many correlated variables} \pause
\end{itemize}

\invisible<1-8>{Predictions will be \alert{variable}}


\end{frame}


\begin{frame}
\frametitle{Mean Square Error}
Suppose $\theta$ is some value of the true parameter \pause \\
\invisible<1>{Bias: \\} \pause
\begin{eqnarray}
\invisible<1-2>{\text{Bias} & = & E[\widehat{\theta} - \theta]\nonumber } \pause
\end{eqnarray}

\invisible<1-3>{We may care about average distance from truth} \pause

\begin{eqnarray}
\invisible<1-4>{\text{E}[(\hat{\theta} - \theta)^{2}]}\pause\invisible<1-5>{ & = & E[\hat{\theta}^{2}]  - 2 \theta E[\hat{\theta}] + \theta^2 } \pause \nonumber \\
 \invisible<1-6>{& = &  E[\hat{\theta}^{2}] - E[\hat{\theta}]^{2} + E[\hat{\theta}]^{2}- 2 \theta E[\hat{\theta}] + \theta^2} \pause  \nonumber \\
\invisible<1-7>{& = & E[\hat{\theta}^{2}] - E[\hat{\theta}]^{2} +  (E[\widehat{\theta} - \theta])^2 } \pause \nonumber \\
  \invisible<1-8>{& = & \text{Var}(\theta) + \text{Bias}^{2} } \pause \nonumber
\end{eqnarray}

\invisible<1-9>{To reduce MSE, we are willing to induce bias to decrease variance$\leadsto$ methods that \alert{shrink} coefficeints toward zero}

\end{frame}


\begin{frame}
\frametitle{Ridge Regression}

Penalty for model complexity \pause

\begin{eqnarray}
\invisible<1>{f(\boldsymbol{\beta}, \boldsymbol{X}, \boldsymbol{Y} ) }\pause \invisible<1-2>{& = & \sum_{i=1}^{N} \left(y_{i} - \
\beta_{0} - \sum_{j=1}^{J}\beta_{j} x_{ij}\right)^{2} } \pause \invisible<1-3>{ + \underbrace{\lambda \sum_{j=1}^{J} \beta_{j}^{2}}_{\text{Penalty}} } \pause \nonumber
\end{eqnarray}

\invisible<1-4>{where:} \pause

\begin{itemize}
\invisible<1-5>{\item[-] $\beta_{0}\leadsto$ intercept} \pause
\invisible<1-6>{\item[-] $\lambda\leadsto$ penalty parameter} \pause
\invisible<1-7>{\item[-] Standardized $\boldsymbol{X}$ (coefficients on same scale)}
\end{itemize}


\end{frame}

\begin{frame}
\frametitle{Ridge Regression$\leadsto$ Optimization}

\begin{eqnarray}
\boldsymbol{\beta}^{\text{Ridge}} & = & \text{arg min}_{\boldsymbol{\beta}} \left\{f(\boldsymbol{\beta}, \boldsymbol{X}, \boldsymbol{Y})\right\} \nonumber  \pause \\
\invisible<1>{& = & \text{arg min}_{\boldsymbol{\beta}} \left\{\sum_{i=1}^{N} \left(y_{i} - \beta_{0} - \sum_{j=1}^{J}\beta_{j} x_{ij}\right)^{2}  + \lambda \sum_{j=1}^{J} \beta_{j}^{2}\right\} } \pause \nonumber \\
 \invisible<1-2>{& = & \text{arg min}_{\boldsymbol{\beta}} \left\{ (\boldsymbol{Y} - \boldsymbol{X}^{'} \boldsymbol{\beta})^{'}(\boldsymbol{Y} - \boldsymbol{X}^{'} \boldsymbol{\beta}) + \lambda \boldsymbol{\beta}^{'}\boldsymbol{\beta} \right\} } \nonumber \\
\invisible<1-3>{& = & \left(\boldsymbol{X}^{'}\boldsymbol{X} + \lambda \boldsymbol{I}_{J}     \right)^{-1} \boldsymbol{X}^{'} \boldsymbol{Y} } \nonumber
\end{eqnarray}

\invisible<1-2>{Demean the data and set $\beta_{0} = \bar{y} = \sum_{i=1}^{N} \frac{y_{i}}{N}$ }
\pause \pause
\end{frame}


\begin{frame}
\frametitle{Ridge Regression$\leadsto$ Intuition (1)}


Suppose $\boldsymbol{X}^{'}\boldsymbol{X} = \boldsymbol{I}_{J}$.  \pause
\begin{eqnarray}
\invisible<1>{\widehat{\boldsymbol{\beta}} & = & \left(\boldsymbol{X}^{'}\boldsymbol{X}\right)^{-1} \boldsymbol{X}^{'}\boldsymbol{Y} \nonumber} \pause  \\
 \invisible<1-2>{& = & \boldsymbol{X}^{'}\boldsymbol{Y} } \pause \nonumber \\
 \invisible<1-3>{\boldsymbol{\beta}^{\text{ridge}} & = & \left(\boldsymbol{X}^{'}\boldsymbol{X} + \lambda \boldsymbol{I}_{J}     \right)^{-1} \boldsymbol{X}^{'} \boldsymbol{Y} \nonumber } \pause \\
  \invisible<1-4>{&= & \left(\boldsymbol{I}_{j} + \lambda \boldsymbol{I}_{j} \right)^{-1} \boldsymbol{X}^{'}\boldsymbol{Y} \nonumber} \pause  \\
   \invisible<1-5>{&= & \left(\boldsymbol{I}_{j} + \lambda \boldsymbol{I}_{j} \right)^{-1} \widehat{\boldsymbol{\beta}} \nonumber } \pause \\
 \invisible<1-6>{\beta_{j}^{\text{Ridge}} & =  & \frac{\widehat{\beta}_{j}}{1 + \lambda} \nonumber }
\end{eqnarray}

\end{frame}

\begin{frame}
\frametitle{Ridge Regression$\leadsto$ Intuition (2)}

\begin{eqnarray}
\boldsymbol{\beta}_{j} & \sim & \text{Normal}(0, \tau^{2}) \nonumber \\
y_{i} & \sim & \text{Normal}(\beta_{0} + \boldsymbol{x}_{i}^{'}\boldsymbol{\beta}, \sigma^{2}) \nonumber
\end{eqnarray}


\pause
\begin{small}
\begin{eqnarray}
\invisible<1>{p(\boldsymbol{\beta}| \boldsymbol{X}, \boldsymbol{Y})  & \propto & \prod_{j=1}^{J} p(\beta_{j}) \prod_{i=1}^{N} p(y_{i}| \boldsymbol{x}_{i}, \boldsymbol{\beta}) \nonumber} \pause  \\
\invisible<1-2>{& \propto &   \prod_{j=1}^{J}\frac{1}{\sqrt{2 \pi} \tau } \exp\left( - \frac{\beta_{j}^2}{2 \tau^2 }  \right) \prod_{i=1}^{N} \frac{1}{\sqrt{2\pi} \sigma} \exp\left( - \frac{ (y_{i} - \beta_{0} - \boldsymbol{x}^{'}_{i} \boldsymbol{\beta})^{2}  }{2 \sigma^2 }   \right) } \nonumber
\end{eqnarray}
\end{small}

\end{frame}


\begin{frame}
\frametitle{Ridge Regression$\leadsto$ Intuition (2)}

\begin{eqnarray}
\log p(\boldsymbol{\beta}| \boldsymbol{X}, \boldsymbol{Y}) & = &  - \sum_{j=1}^{J} \frac{\beta_{j}^2}{2 \tau^2 } - \sum_{i=1}^{N} \frac{ (y_{i} - \beta_{0} - \boldsymbol{x}^{'} \boldsymbol{\beta})^{2}  }{2 \sigma^2 } \nonumber \pause \\
\invisible<1>{- 2 \sigma^2\log p(\boldsymbol{\beta}| \boldsymbol{X}, \boldsymbol{Y}) & = &   \sum_{i=1}^{N} (y_{i} - \beta_{0} - \boldsymbol{x}^{'} \boldsymbol{\beta})^{2} + \sum_{j=1}^{J} \frac{\sigma^2}{\tau^2} \beta_{j}^2  } \pause  \nonumber
\end{eqnarray}

\invisible<1-2>{where:} \pause
\begin{itemize}
\invisible<1-3>{\item[-] $\lambda = \frac{\sigma^2}{\tau^2} $}
\end{itemize}

\end{frame}


\begin{frame}
\frametitle{Ridge Regression $\leadsto$ Intuition (3) }

\begin{defn}
Suppose $\boldsymbol{X}$ is an $N \times J$ matrix.  Then $\boldsymbol{X}$ can be written as:

\begin{eqnarray}
\boldsymbol{X} & =& \underbrace{\boldsymbol{U}}_{N \times N} \underbrace{\boldsymbol{S}}_{N \times J} \underbrace{\boldsymbol{V}^{'}}_{J \times J} \nonumber
\end{eqnarray}

Where:
\begin{eqnarray}
\boldsymbol{U}^{'}\boldsymbol{U} & = & \boldsymbol{I}_{N} \nonumber \\
\boldsymbol{V}^{'}\boldsymbol{V} & = & \boldsymbol{V}\boldsymbol{V}^{'} = \boldsymbol{I}_{J} \nonumber
\end{eqnarray}
$\boldsymbol{S}$ contains $\min(N, J)$ singular values, $\sqrt{\lambda_{j}}\geq 0$ down the diagonal and then 0's for the remaining entries
\end{defn}


\end{frame}

\begin{frame}
\frametitle{Ridge Regression $\leadsto$ Intuition (3) }

\begin{small}
Recall: PCA:
\begin{eqnarray}
\frac{1}{N} \boldsymbol{X}^{'}\boldsymbol{X} = \underbrace{\boldsymbol{W}}_{\text{eigenvectors}} \begin{pmatrix} \lambda_{1} & 0 & \hdots & 0 \\
			0 		& \lambda_{2} & \hdots & 0 \\
			\vdots	& \vdots     & \ddots & \vdots \\
			0       & 0 		& \hdots   & \lambda_{J}\\
			\end{pmatrix}
			\underbrace{ \boldsymbol{W}^{'}}_{\text{eigenvectors}}   \nonumber
			\end{eqnarray} \pause

\invisible<1>{Using SVD:} \pause

\begin{eqnarray}
\invisible<1-2>{\frac{1}{N} \boldsymbol{X}^{'}\boldsymbol{X} & = & \boldsymbol{V} \boldsymbol{S}^{'}\underbrace{\left(\boldsymbol{U}^{'}\boldsymbol{U}\right)}_{\boldsymbol{I}_{J}} \boldsymbol{S} \boldsymbol{V}^{'} \nonumber } \pause \\
\invisible<1-3>{&  = & \frac{1}{N} \boldsymbol{V} \boldsymbol{S}^{'} \boldsymbol{S} \boldsymbol{V}^{'} \nonumber } \pause \\
\invisible<1-4>{& = &   \underbrace{\boldsymbol{V}}_{\text{eigenvectors}} \begin{pmatrix} \lambda_{1} & 0 & \hdots & 0 \\
			0 		& \lambda_{2} & \hdots & 0 \\
			\vdots	& \vdots     & \ddots & \vdots \\
			0       & 0 		& \hdots   & \lambda_{J}\\
			\end{pmatrix} \underbrace{\boldsymbol{V}^{'}}_{\text{eigenvectors}} \nonumber }
\end{eqnarray}

\end{small}

\end{frame}


\begin{frame}
\frametitle{Ridge Regression $\leadsto$ Intuition (3) }
\begin{footnotesize}
We can write the predicted values for a regular regression as
\begin{eqnarray}
\hat{Y} & = & \boldsymbol{X} \hat{\boldsymbol{\beta}} \nonumber \\
		& = & \boldsymbol{X} \left(\boldsymbol{X}^{'}\boldsymbol{X} \right)^{-1}\boldsymbol{X}^{'} \boldsymbol{Y} \nonumber \\
		& = & \boldsymbol{U} \boldsymbol{U}^{'}\boldsymbol{Y}  = \sum_{j=1}^{J} \boldsymbol{u}_{j} \boldsymbol{u}_{j}^{'} \boldsymbol{Y} \nonumber
\end{eqnarray}
\pause

\invisible<1>{We can write $\boldsymbol{\beta}^{\text{ridge}}$ as} \pause
\begin{eqnarray}
\invisible<1-2>{\hat{Y}^{\text{ridge}}& = & \boldsymbol{X}\left(\boldsymbol{X}^{'}\boldsymbol{X} + \lambda \boldsymbol{I}_{J}    \right)^{-1}\boldsymbol{X}^{'}\boldsymbol{Y} \nonumber } \pause \\
\invisible<1-3>{& = & \boldsymbol{U} \tilde{\boldsymbol{S}} \boldsymbol{U}^{'}\boldsymbol{Y} \nonumber }
\end{eqnarray}



\invisible<1-3>{Where
\begin{eqnarray}
\tilde{\boldsymbol{S}} &= & \left[\boldsymbol{S}(\boldsymbol{S}^{'}\boldsymbol{S} + \lambda \boldsymbol{I}_{J})^{-1} \boldsymbol{S} \right] \nonumber
\end{eqnarray}}

\pause

\invisible<1-4>{Which we can write as:
\begin{eqnarray}
\hat{Y}^{\text{ridge}}& = & \sum_{j=1}^{J} \boldsymbol{u}_{j} \frac{\lambda_{j}}{\lambda_{j} + \lambda} \boldsymbol{u}_{j}^{'} \boldsymbol{Y} \nonumber
\end{eqnarray}}


\end{footnotesize}

\end{frame}


\begin{frame}
\frametitle{Degrees of Freedom for Ridge}

We will say that the degrees of freedom for Ridge regression with penalty $\lambda$ is
\begin{eqnarray}
\text{dof}(\lambda ) & = & \sum_{j=1}^{J} \frac{\lambda_{j}}{\lambda_{j} + \lambda} \nonumber
\end{eqnarray}


\end{frame}





\begin{frame}
\frametitle{Lasso Regression Objective Function}

Different Penalty for Model Complexity

\begin{eqnarray}
f(\boldsymbol{\beta}, \boldsymbol{X}, \boldsymbol{Y} ) & = & \sum_{i=1}^{N} \left(y_{i} - \beta_{0} - \sum_{j=1}^{J} \beta_{j} x_{ij}  \right)^{2} + \lambda \sum_{j=1}^{J} \underbrace{|\beta_{j}|}_{\text{Penalty}} \nonumber \pause
\end{eqnarray}



\end{frame}


\begin{frame}
\frametitle{Lasso Regression Optimization}

\begin{defn}
\alert{Coordinate Descent Algorithms: } \\
Consider $g:\Re^{J} \rightarrow \Re$.  Our goal is to find $\boldsymbol{x}^{*} \in \Re^{J}$ such that $g(\boldsymbol{x}^{*}) \leq g(\boldsymbol{x})$ for all $\boldsymbol{x} \in \Re$. \\

To find $\boldsymbol{x}^{*}$:


Until convergence: for each iteration $t$ and each coordinate $j$

\begin{eqnarray}
x_{j}^{t + 1} & = & \text{arg min}_{x_{j} \in \Re}g(x_{1}^{t +1}, x_{2}^{t + 1}, \hdots, x_{j-1}^{t+1}, x_{j}, x_{j+1}^{t}, \hdots, x_{J}^{t}) \nonumber
\end{eqnarray}

\end{defn}

\end{frame}



\begin{frame}
\frametitle{Lasso Regression Optimization: Coordinate Descent}


\begin{eqnarray}
 \tilde{f}(\boldsymbol{\beta}, \boldsymbol{X}, \boldsymbol{Y} ) & = & \frac{1}{2N} \sum_{i=1}^{N} \left(y_{i} - \beta_{0} - \sum_{j=1}^{J} \beta_{j} x_{ij}  \right)^{2} + \lambda \sum_{j=1}^{J} |\beta_{j}| \nonumber \pause
\end{eqnarray}

\begin{itemize}
\invisible<1>{\item[-] \alert{Case 1}: If $\beta_{j} = 0 \leadsto $ not differentiable. But $\beta_{j} = 0$} \pause
\invisible<1-2>{\item[-] \alert{Case 2}: If $ \beta_{j}>(<) 0 \leadsto$ differentiable $\leadsto$ differentiate and solve for $\beta_{j}$ } \pause
\end{itemize}

\invisible<1-3>{Define $\tilde{y}_{i}^{j} = \beta_{0} + \sum_{l \neq j} x_{il} \beta_{l} $ \\} \pause

\invisible<1-4>{$r^{j} \equiv  \frac{1}{N} \sum_{i=1}^{N} x_{ij}(y_{i} - \tilde{y}_{i}^{j} )$} \pause


\invisible<1-5>{Update step for $\beta_{j}$ is } \pause

\begin{eqnarray}
\invisible<1-6>{\beta_{j} & \leftarrow  & \text{sign}(r^{j})\text{max}(|r^{j}| - \lambda, 0 ) \nonumber }
\end{eqnarray}




\end{frame}






\begin{frame}
\frametitle{Lasso Regression$\leadsto$ Intuition 1, Soft Thresholding}

Suppose again $\boldsymbol{X}^{'}\boldsymbol{X} = \boldsymbol{I}_{J}$ \pause

\begin{eqnarray}
\invisible<1>{f(\boldsymbol{\beta}, \boldsymbol{X}, \boldsymbol{Y} ) & = & \left(Y - \boldsymbol{X}\boldsymbol{\beta} \right)^{'}\left(Y - \boldsymbol{X}\boldsymbol{\beta} \right)  + \lambda \sum_{j=1}^{J}| \beta_{j}| \nonumber \\} \pause
 \invisible<1-2>{& = & - 2 \boldsymbol{X}^{'}\boldsymbol{Y} \boldsymbol{\beta} + \boldsymbol{\beta}^{'}\boldsymbol{\beta} + \lambda  \sum_{j=1}^{J}| \beta_{j}| } \pause  \nonumber
\end{eqnarray}

\invisible<1-3>{The coefficient is } \pause
\begin{eqnarray}
\invisible<1-4>{\beta_{j}^{\text{LASSO}} & = & \text{sign}\left(\widehat{\beta}_{j}\right) \left(|\widehat{\beta}_{j}| - \lambda  \right)_{+} \nonumber } \pause
\end{eqnarray}

\begin{itemize}
\invisible<1-5>{\item[-] $\text{sign}(\cdot) \leadsto$ $1$ or $-1$} \pause
\invisible<1-6>{\item[-] $\left( |\widehat{\beta}_{j}| - \lambda \right)_{+} = \text{max}( |\widehat{\beta}_{j}| - \lambda, 0 )$}
\end{itemize}

\end{frame}

\begin{frame}
\frametitle{Lasso Regression$\leadsto$ Intuition 1, Soft Thresholding}


Compare soft assignment \pause
\begin{eqnarray}
\invisible<1>{\beta_{j}^{\text{LASSO}} & = & \text{sign}\left(\widehat{\beta}_{j}\right) \left(|\widehat{\beta}_{j}| - \lambda  \right)_{+} } \pause \nonumber
\end{eqnarray}

\invisible<1-2>{With hard assignment, selecting $M$ biggest components} \pause
\begin{eqnarray}
\invisible<1-3>{\beta_{j}^{\text{subset}} & = & \widehat{\beta}_{j} \cdot I\left(|\widehat{\beta}_{j}| \geq | \widehat{\beta}_{(M)} |     \right) \nonumber } \pause
\end{eqnarray}


\invisible<1-4>{Intuition 2: Prior on coefficients $\leadsto$ Laplace ``The Bayesian LASSO" } \pause

\invisible<1-5>{Why does LASSO induce sparsity?}

\end{frame}



\begin{frame}
\frametitle{Comparing Ridge and LASSO}


\only<1>{\scalebox{0.8}{\includegraphics{RidgeExamp1.pdf}}}
\only<2>{\scalebox{0.8}{\includegraphics{LassoExamp1.pdf}}}


\end{frame}

\begin{frame}
\frametitle{Comparing Ridge and LASSO}

Contrast $\beta = (\frac{1}{\sqrt{2}},\frac{1}{\sqrt{2}} )$ and $\tilde{\beta} = (1, 0)$ \pause

\invisible<1>{Under ridge:}\pause
\begin{eqnarray}
\invisible<1-2>{\sum_{j=1}^{2} \beta_{j}^{2} & = & \frac{1}{2} + \frac{1}{2} = 1\nonumber \\} \pause
\invisible<1-3>{\sum_{j=1}^{2} \tilde{\beta}_{j}^{2}  & = &  1 + 0 = 1 } \pause \nonumber
\end{eqnarray}

\invisible<1-4>{Under LASSO } \pause
\begin{eqnarray}
\invisible<1-5>{\sum_{j=1}^{2} |\beta_{j}| & = & \frac{1}{\sqrt{2}} + \frac{1}{\sqrt{2}}  = \sqrt{2} \nonumber \\} \pause
\invisible<1-6>{\sum_{j=1}^{2} |\tilde{\beta}_{j}| & = & 1 +0 = 1 \nonumber }
\end{eqnarray}

\end{frame}


\begin{frame}
\frametitle{Ridge and LASSO: The Elastic-Net}

Combining the two criteria $\leadsto$ Elastic-Net

\begin{small}
\begin{eqnarray}
f(\boldsymbol{\beta}, \boldsymbol{X}, \boldsymbol{Y} ) & = & \frac{1}{N} \sum_{i=1}^{N}\left(y_{i} - \beta_{0} - \sum_{j=1}^{J} \beta_{j} x_{ij} \right)^2 + \lambda \sum_{j=1}^{J} \left(\frac{1}{2} (1-\alpha)\beta_{j}^2 + \alpha|\beta_{j}|    \right) \nonumber
\end{eqnarray}
\end{small}

\pause

\invisible<1>{The new update step (for coordinate descent:)} \pause

\invisible<1-2>{\begin{eqnarray}
\beta_{j} & \leftarrow & \frac{\text{sign}(r^{j})\text{max}(|r^{j}| - \lambda \alpha, 0)}{1 + \lambda (1- \alpha) } \nonumber
\end{eqnarray}
}
\end{frame}




\begin{frame}
\frametitle{Selecting $\lambda$}

How do we determine $\lambda$? $\leadsto$ Cross validation (Recall code from Monday) \pause \\
\invisible<1>{Applying models gives score (probability) of document belong to class$\leadsto$ threshold to classify} \pause \\


\end{frame}


\begin{frame}
\frametitle{Credit Claiming (Grimmer, Westwood, and Messing 2014)}

\begin{footnotesize}
%\#\#credit is a 797 element long binary vector

%\#\#dtm is a 797 x 7587 document term matrix
\begin{semiverbatim}
\only<1>{library(glmnet)

set.seed(8675309) \#\#setting seed

folds<- sample(1:10, nrow(dtm), replace=T) \#\#assigning to fold

out\_of\_samp<- c()  \#\#collecting the predictions}


\only<2>{for(z in 1:10)\{

	train<- which(folds!=z) \#\#the observations we will use to train the model

	test<- which(folds==z) \#\#the observations we will use to test the model

	part1<- cv.glmnet(x = dtm[train,], y = credit[train], alpha = 1, family = \'binomial\') \#\#fitting the LASSO model on the data.

	\#\# alpha = 1 -> LASSO

	\#\# alpha = 0 -> RIDGE

	\#\# 0<alpha<1 -> Elastic-Net

	out\_of\_samp[test]<- predict(part1, newx= dtm[test,], s = part1\$lambda.min, type =\'class\') \#\#predicting the labels

	print(z) \#\#printing the labels

	\}

conf\_table<- table(out\_of\_samp, credit)	\#\#calculating the confusion table

> round(sum(diag(conf\_table))/len(credit), 3)

[1] \alert{0.844}
}
\end{semiverbatim}
\end{footnotesize}
\end{frame}



\begin{frame}
\frametitle{Generalized Cross Validation and Ridge Regression}

In some special cases there are analytic solutions: \\ \pause

\begin{eqnarray}
\invisible<1>{\boldsymbol{\beta}^{\text{Ridge}} & = & \left(\boldsymbol{X}^{'}\boldsymbol{X} + \lambda \boldsymbol{I}_{J} \right)^{-1} \boldsymbol{X}^{'} \boldsymbol{Y} \nonumber } \pause \\
\invisible<1-2>{\widehat{\boldsymbol{Y}} & = & \boldsymbol{X}(\boldsymbol{\beta})^{\text{Ridge}} \nonumber\\} \pause
\invisible<1-3>{& = & \underbrace{\boldsymbol{X}\left(\boldsymbol{X}^{'}\boldsymbol{X} + \lambda \boldsymbol{I}_{J} \right)^{-1} \boldsymbol{X}^{'}}_{\text{Hat Matrix}} \boldsymbol{Y} \nonumber\\} \pause
\invisible<1-4>{\widehat{\boldsymbol{Y}} & = & \underbrace{\boldsymbol{H}}_{\text{Smoother Matrix}} \boldsymbol{Y}  \nonumber } \pause
\end{eqnarray}


\end{frame}

\begin{frame}
\frametitle{Generalized Cross Validation and Ridge Regression}


Why do we care?  \pause \\
\invisible<1>{Leave one out cross validation\\} \pause
\begin{eqnarray}
\invisible<1-2>{\text{Cross Validation}(1) & = & \frac{1}{N} \sum_{i=1}^{N} (Y_{i} - f (\boldsymbol{X}_{-i}, \boldsymbol{Y}_{-i}, \lambda, \hat{\boldsymbol{\beta}} ))^{2} \nonumber \\} \pause
\invisible<1-3>{& = & \frac{1}{N} \sum_{i=1}^{N}  \left(\frac{Y_{i} - f (\boldsymbol{X}, \boldsymbol{Y}, \lambda, \hat{\boldsymbol{\beta} }) }{1 - H_{ii}} \right)^2 \nonumber}
\end{eqnarray}



\end{frame}

\begin{frame}
\frametitle{Generalized Cross Validation and Ridge Regression}

Calculating $\boldsymbol{H}$ can be computationally expensive \pause \\
\begin{itemize}
\invisible<1>{\item[-] $\text{Trace}(\boldsymbol{H}) \equiv \text{Tr}(\boldsymbol{H})  = \sum_{i=1}^{N} H_{ii} $ } \pause
\invisible<1-2>{\item[-] $\text{Tr}(\boldsymbol{H})$ = Effective number of parameters (class regression = number of independent variables + 1)} \pause
\invisible<1-3>{\item[-] For Ridge regression:} \pause
\begin{eqnarray}
\invisible<1-4>{\text{Tr}(\boldsymbol{H}) & = & \sum_{j=1}^{J} \frac{\lambda_{j}}{\lambda_{j} + \underbrace{\lambda}_{\text{Penalty}}} \nonumber } \pause
\end{eqnarray}
\invisible<1-5>{where $\lambda_{j}$ is the $j^{\text{th}}$ Eigenvalue from $\boldsymbol{\Sigma} = \boldsymbol{X}^{'}\boldsymbol{X}$} \pause \invisible<1-6>{ (!!!!!)} \pause
\end{itemize}


\invisible<1-7>{Define generalized cross validation:} \pause
\begin{eqnarray}
\invisible<1-8>{\text{GCV} &  = & \frac{1}{N} \sum_{i=1}^{N} \left( \frac{Y_{i} - \hat{Y}_{i}}{1 - \frac{\text{Tr}(\boldsymbol{H})}{N} }    \right)^2 \nonumber } \pause
\end{eqnarray}

\invisible<1-9>{Applicable in any setting where we can write \alert{Smoother} matrix} \pause

\end{frame}




\end{document}
