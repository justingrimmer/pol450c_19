\documentclass{beamer}

%\usepackage[table]{xcolor}
\mode<presentation> {
  \usetheme{Boadilla}
%  \usetheme{Pittsburgh}
%\usefonttheme[2]{sans}
\renewcommand{\familydefault}{cmss}
%\usepackage{lmodern}
%\usepackage[T1]{fontenc}
%\usepackage{palatino}
%\usepackage{cmbright}
  \setbeamercovered{transparent}
\useinnertheme{rectangles}
}
%\usepackage{normalem}{ulem}
%\usepackage{colortbl, textcomp}
\setbeamercolor{normal text}{fg=black}
\setbeamercolor{structure}{fg= black}
\definecolor{trial}{cmyk}{1,0,0, 0}
\definecolor{trial2}{cmyk}{0.00,0,1, 0}
\definecolor{darkgreen}{rgb}{0,.4, 0.1}
\usepackage{array}
\beamertemplatesolidbackgroundcolor{white}  \setbeamercolor{alerted
text}{fg=red}

\setbeamertemplate{caption}[numbered]\newcounter{mylastframe}

%\usepackage{color}
\usepackage{tikz}
\usetikzlibrary{arrows}
\usepackage{colortbl}
%\usepackage[usenames, dvipsnames]{color}
%\setbeamertemplate{caption}[numbered]\newcounter{mylastframe}c
%\newcolumntype{Y}{\columncolor[cmyk]{0, 0, 1, 0}\raggedright}
%\newcolumntype{C}{\columncolor[cmyk]{1, 0, 0, 0}\raggedright}
%\newcolumntype{G}{\columncolor[rgb]{0, 1, 0}\raggedright}
%\newcolumntype{R}{\columncolor[rgb]{1, 0, 0}\raggedright}

%\begin{beamerboxesrounded}[upper=uppercol,lower=lowercol,shadow=true]{Block}
%$A = B$.
%\end{beamerboxesrounded}}
\renewcommand{\familydefault}{cmss}
%\usepackage[all]{xy}

\usepackage{tikz}
\usepackage{lipsum}

 \newenvironment{changemargin}[3]{%
 \begin{list}{}{%
 \setlength{\topsep}{0pt}%
 \setlength{\leftmargin}{#1}%
 \setlength{\rightmargin}{#2}%
 \setlength{\topmargin}{#3}%
 \setlength{\listparindent}{\parindent}%
 \setlength{\itemindent}{\parindent}%
 \setlength{\parsep}{\parskip}%
 }%
\item[]}{\end{list}}
\usetikzlibrary{arrows}
%\usepackage{palatino}
%\usepackage{eulervm}
\usecolortheme{lily}
\newtheorem{ass}{Assumption}
\newtheorem{com}{Comment}
\newtheorem{lem} {Lemma}
\newtheorem{prop}{Proposition}
\newtheorem{thm}{Theorem}
\newtheorem{defn}{Definition}
\newtheorem{cor}{Corollary}
\newtheorem{obs}{Observation}
 \numberwithin{equation}{section}

\title[Methodology III] % (optional, nur bei langen Titeln nötig)
{Political Methodology III: Model Based Inference}

\author[]{Justin Grimmer \\
(Jens Hainmueller Based Slides)}

\date{May 9th, 2019}

\renewcommand\r{\right}
\renewcommand\l{\left}
\definecolor{dgreen}{rgb}{0,.6,0.8}
\newtheorem{assumption}{Assumption}
\newtheorem{iass}{Identification Assumption}
\newtheorem{ires}{Identification Result}
\newtheorem{estm}{Estimand}
\newtheorem{esti}{Estimator}
\newtheorem{question}{Question}
\newtheorem{answer}{Answer}
\newtheorem{proposition}{Proposition}
\newcommand{\bs}{\boldsymbol}
\newcommand{\mb}{\mathbf}
\newcommand{\real}{\mathbb{R}}
\renewcommand{\u}{\ensuremath{\mb{u}}}
\renewcommand{\v}{\ensuremath{\mb{v}}}
\newcommand{\U}{\ensuremath{\mb{U}}}
\newcommand{\M}{\ensuremath{\mb{M}}}
\newcommand{\X}{\ensuremath{\mb{X}}}
\newcommand{\x}{\ensuremath{\mb{x}}}
\newcommand{\y}{\ensuremath{\mb{y}}}
\renewcommand{\b}{\ensuremath{\bs{\beta}}}
\newcommand{\e}{\ensuremath{\bs{\epsilon}}}
\newcommand{\bhat}{\ensuremath{\hat{\bs{\beta}}}}
\newcommand{\XX}{\ensuremath{\X'\X}}
\newcommand{\XXinv}{\ensuremath{\left(\XX\right)^{-1}}}
\newcommand{\hatsig}{\ensuremath{\hat{\sigma}^2}}
\newcommand{\sig}{\ensuremath{\sigma^2}}
\newcommand{\hatmat}{\ensuremath{\X \XXinv \X'}}
\newcommand{\ehat}{\ensuremath{\hat{\e}}}
\newcommand{\tr}{\text{tr}}
\newcommand{\Sig}{\ensuremath{\bs{\Sigma}}}
\newcommand{\SigInv}{\ensuremath{\bs{\Sigma^{-1}}}}
\newcommand{\W}{\ensuremath{\mb{W}}}
\newcommand{\cN}{\mathcal{N}}
\newcommand\spacingset[1]{\renewcommand{\baselinestretch}{#1}\small\normalsize}
\newcommand{\var}{\text{Var}}
\newcommand{\E}{\mathbb{E}}
\newcommand{\cov}{\mbox{Cov}}
\newcommand{\Cov}{\mbox{Cov}}
%\newcommand{\cor}{\mbox{Cor}}
\newcommand{\avepop}{\frac{1}{n}\sum_{i=1}^n}
\newcommand{\indep}{{\bot\negthickspace\negthickspace\bot}}

\DeclareMathOperator*\plim{plim}
\useoutertheme{infolines}

\setbeamertemplate{navigation symbols}{}

\begin{document}


\subject{Talks}
\AtBeginSection[]
{
  \begin{frame}[plain]
    \frametitle{}
    \tableofcontents[currentsection]
    \addtocounter{framenumber}{-1}
  \end{frame}
}

\AtBeginSubsection[]
{
  \begin{frame}[plain]
    \frametitle{}
    \tableofcontents[currentsection,currentsubsection]
    \addtocounter{framenumber}{-1}
  \end{frame}
}


\frame[plain]{
  \titlepage
  \addtocounter{framenumber}{-1}
}

%%%%%%%%%%%%%%%%%


\section{Learning the Regression Function}


 \begin{frame}[fragile]
\frametitle{Linear Models}

Recall the linear model,
$$
y_i = \beta_0 + x_{1i}\beta_1 + x_{2i}\beta_2 + x_{3i}\beta_3 + u_i
$$\pause

Advantages:
\begin{itemize}
\item Simplicity
\item Interpretability
\item Easy to do inference
\end{itemize}
\medskip
Downsides:
\begin{itemize}
\item Functional form assumptions (Linearity and Additivity)
\item No learning
\end{itemize}

\end{frame}




\frame{
\frametitle{Nonlinearity}

Linearity of the Conditional Expectation Function (\alert{$\y = \X \b +\mb{u}$}) is a key assumption. Why? \pause
\begin{figure}[ht] \centering
        \scalebox{.35}{\includegraphics{fig1-nl.pdf}}
     \end{figure}

}

\frame{
\frametitle{Nonlinearity}

Linearity of the Conditional Expectation Function (\alert{$\y = \X \b +\mb{u}$}) is a key assumption. Why?
\begin{figure}[ht] \centering
        \scalebox{.35}{\includegraphics{fig2-nl.pdf}}
     \end{figure}

}


\frame{
\frametitle{Nonlinearity}
    \begin{itemize}
        \item If $E[Y|\X]$ is not linear in $\X$, $E[\mb{u}|\X] \ne 0$ for all values $\X=\mb{x}$ and $\hat{\b}$ may be biased and inconsistent.
        \bigskip\pause
  \item Nonlinearities may be important but few social scientific theories offer any guidance as to functional form whatsoever. \pause
\begin{itemize}
  \item Statements like ``y increases with x'' (monotonicity)
are as specific as most social theories get.\medskip
  \item Possible Exceptions: Returns to scale, constant elasticities, interactive effects, cyclical patterns in time series data, etc.\pause
\end{itemize}\bigskip
  \item Usually we employ ``linearity by default'' but we should try to make sure this is appropriate: \alert{detect} non-linearities and \alert{model} them accurately
\end{itemize}

}

\begin{frame}[fragile]

\frametitle{Diagnosing Nonlinearity}

\begin{itemize}
\item For \alert{marginal} relationships $Y$ and $X$
\begin{itemize}
  \item Scatterplots with loess lines
\end{itemize}\bigskip \pause
\item For \alert{partial} relationships $Y$ and $X_1$, controlling for $X_2$, $X_3$,...,$X_k$ the regression surface is high-dimensional. We need other diagnostic tools such as:
\begin{itemize}
  \item Added variables plots and component residual plots 
  \end{itemize}\bigskip
  \item Model diagnostics become rather difficult once we have many predictors
%  \item \alert{Alternative}: Use a learning model that learns the functional form from the data (within a space of functions)
%     \begin{itemize}
 %     \item Semi-parametric regression techniques: \alert{Generalized Additive Models} (GAMs) 
 %   \item Other machine learning methods: Random forrest, boosted trees, ridge/lasso/elastic net regression, kernel based methods, etc.
%\end{itemize}
%  \item loess plots of $Y$ and $X$s
%  \item added variable plots (plots of $Y$ and $X_j$ after ``controlling'' for linear component of other $X$s)
%  \item component plus residual plots (aka partial \emph{residual}
%  plots)
%  \item
% \item
\end{itemize}

\end{frame}

\frame{
\frametitle{How should we deal with nonlinearity?}
\small
Given we have a linear regression model, our options are somewhat limited. We can partially address nonlinearity by:
\begin{itemize}
  \item Breaking categorical or continuous variables into dummy variables (e.g. education levels)
  \item Including interactions
  \item Including polynomial terms
  \item Transformations such as logs
\end{itemize}\bigskip

 \alert{Alternative}: Use a learning model that learns the functional form from the data (within a space of functions)
  
     \begin{itemize}
      \item Semi-parametric regression techniques: \alert{Generalized Additive Models} (GAMs) 
    \item Other machine learning methods: random forrest, boosted trees, ridge/lasso/elastic net regression, kernel based methods, etc.
\end{itemize}

}

















%
%\frame{
%\frametitle{Example: Attitudes Towards Immigration}
%
%\begin{itemize}
%  \item Outcome: Pro-Immigration Attitudes
%\begin{footnotesize}
%  \begin{itemize}
%    \item 1 Strongly opposed to increase in immigration
%    \item .
%    \item 5 Strongly in favour of increase in immigration
%  \end{itemize}
%\end{footnotesize}
%  \item Highest Educational Attainment:\\
%\begin{footnotesize}
%\begin{itemize}
%  \item 1 Less than high school
%  \item 2 Some high school, no diploma
%  \item 3 Graduated from high school- Diploma or equivalent (GED)
%  \item 4 Some college, no degree
%  \item 5 Associate degree (AA, AS)
%  \item 6 Bachelor's degree
%  \item 7 Master's degree
%  \item 8 Professional degree (MD, DDS, LLB, JD)
%  \item 9 Doctorate degree
%\end{itemize}
%\end{footnotesize}
%%  \item Household Income
%%\begin{footnotesize}
%%  \begin{itemize}
%%    \item 1  Less than \$5,000
%%    \item
%%    \item 19 More than \$175,000
%%\end{itemize}
%%\end{footnotesize}
%  \item Age of Respondent
%\begin{footnotesize}
%  \begin{itemize}
%    \item 18 years to 93 years
%\end{itemize}
%\end{footnotesize}
%
%\end{itemize}
%
%}
%
%\frame{
%\frametitle{Attitudes Towards Immigration (2008)}
%
%\begin{figure}[ht] \centering
%        \scalebox{.55}{\includegraphics{fig0-nl.pdf}}
%     \end{figure}
%
%}
%
%\frame{
%\frametitle{Immigration Attitudes and Education}
%
%\begin{figure}[ht] \centering
%        \scalebox{.45}{\includegraphics{fig1i-nl.pdf}}
%     \end{figure}
%
%}
%
%\frame{
%\frametitle{Immigration Attitudes and Education}
%
%\begin{figure}[ht] \centering
%        \scalebox{.45}{\includegraphics{fig2i-nl.pdf}}
%     \end{figure}
%
%}
%
%
%\begin{frame}[fragile]
%\frametitle{Immigration Attitudes and Education}
%
%\begin{Verbatim}[fontsize=\footnotesize, frame=single, label=R Code]
%> mod0 <- lm(imgpro5mod1~ppeduc, data=d)
%> coeftest(mod0)
%
%t test of coefficients:
%
%            Estimate Std. Error t value  Pr(>|t|)
%(Intercept) 1.859180   0.081894 22.7022 < 2.2e-16 ***
%ppeduc      0.165658   0.017636  9.3929 < 2.2e-16 ***
%---
%Signif. codes:  0 ‘***’ 0.001 ‘**’ 0.01 ‘*’ 0.05 ‘.’ 0.1 ‘ ’ 1
%\end{Verbatim}
%
%\end{frame}
%
%\begin{frame}[fragile]
%\frametitle{Immigration Attitudes and Education}
%
%\begin{Verbatim}[fontsize=\footnotesize, frame=single, label=R Code]
%> mod0a <- lm(imgpro5mod1~factor(ppeduc), data=d)
%> coeftest(mod0a)
%
%t test of coefficients:
%
%                Estimate Std. Error t value  Pr(>|t|)
%(Intercept)     2.173913   0.253720  8.5682 < 2.2e-16 ***
%factor(ppeduc)2 0.261461   0.272848  0.9583 0.3380747
%factor(ppeduc)3 0.096465   0.259456  0.3718 0.7100945
%factor(ppeduc)4 0.324500   0.262820  1.2347 0.2171310
%factor(ppeduc)5 0.461008   0.275907  1.6709 0.0949432 .
%factor(ppeduc)6 0.673605   0.263864  2.5528 0.0107779 *
%factor(ppeduc)7 1.059170   0.274784  3.8546 0.0001206 ***
%factor(ppeduc)8 0.740373   0.326614  2.2668 0.0235363 *
%factor(ppeduc)9 1.186087   0.351565  3.3737 0.0007595 ***
%\end{Verbatim}
%
%\end{frame}
%
%
%\frame{
%\frametitle{Immigration Attitudes and Age}
%
%\begin{figure}[ht] \centering
%        \scalebox{.45}{\includegraphics{fig5i-nl.pdf}}
%     \end{figure}
%
%}
%
%\frame{
%\frametitle{Immigration Attitudes and Age}
%
%\begin{figure}[ht] \centering
%        \scalebox{.45}{\includegraphics{fig6i-nl.pdf}}
%     \end{figure}
%
%}
%
%\begin{frame}[fragile]
%\frametitle{Immigration Attitudes and Age}
%
%\begin{Verbatim}[fontsize=\footnotesize, frame=single, label=R Code]
%> mod1 <- lm(imgpro5mod1~ppage, data=d)
%> coeftest(mod1)
%t test of coefficients:
%              Estimate Std. Error t value  Pr(>|t|)
%(Intercept)  2.8201106  0.0957797  29.444 < 2.2e-16 ***
%ppage       -0.0051324  0.0018772  -2.734  0.006326 **
%>
%> mod1a <- lm(imgpro5mod1~ppage+I(ppage^2), data=d)
%> coeftest(mod1a)
%t test of coefficients:
%               Estimate  Std. Error t value  Pr(>|t|)
%(Intercept)  3.60493245  0.23929787 15.0646 < 2.2e-16 ***
%ppage       -0.04123139  0.01026562 -4.0165 6.184e-05 ***
%I(ppage^2)   0.00036702  0.00010262  3.5763 0.0003589 ***
%\end{Verbatim}
%
%\end{frame}
%
%\frame{
%\frametitle{Immigration Attitudes, Education and Age}
%
%\begin{figure}[ht] \centering
%        \scalebox{.45}{\includegraphics{fig3d1-nl.pdf}}
%     \end{figure}
%
%}
%
%\frame{
%\frametitle{Immigration Attitudes, Education and Age}
%
%\begin{figure}[ht] \centering
%        \scalebox{.45}{\includegraphics{fig3d2-nl.pdf}}
%     \end{figure}
%}
%
%\subsection{Added Variable Plots}
%
%
%\frame{
%
%\frametitle{Added Variable Plots}
%\small
%AV plots allow to visualize the relationship between $Y$ and $X_j$, after the linear component of the other $X$ variables has been controlled for.\\\bigskip
%
%Constructing an \alert{added variable plot} for $X_j$
%\begin{enumerate}
%\item Get residuals from regressing $Y$ on all predictors but omitting $X_j$\pause
%\item Get residuals from regressing $X_j$ on all other predictors\pause
%\item Plot the residuals from (1) against the residuals from (2)\pause
%\begin{itemize}
%\item Note that these steps are identical to the ``partialing out'' procedure
%\item In R: \texttt{avPlots(model)} from \texttt{car} package
%\end{itemize}
%\end{enumerate}\bigskip\pause
%
%Facts about AV Plots
%\begin{itemize}
%  \item LS fit to the AV-plot has slope $\hat{\beta}_j$ and intercept zero.\pause
%  \item Residuals from LS fit to the AV-plot are identical to the residuals from LS fit of the original model ($Y$ on all $X$ including $X_j$).\pause
%  \item Add Loess fit to detect various violations such as nonlinearity, heteroscedasticity, or outliers.
%\end{itemize}
%
%}
%
%
%\begin{frame}[fragile]
%\frametitle{Added Variable Plot for Age}
%
%\begin{Verbatim}[fontsize=\footnotesize, frame=single, label=R Code, commandchars=\\\{\}]
%> y_educ <-lm(imgpro5mod1~ppeduc,data=d) # \alert{1: Y on all but Xj}
%> coeftest(y_educ)
%t test of coefficients:
%            Estimate Std. Error t value  Pr(>|t|)
%(Intercept) 1.859180   0.081894 22.7022 < 2.2e-16 ***
%ppeduc      0.165658   0.017636  9.3929 < 2.2e-16 ***
%
%> age_educ <-lm(ppage~ppeduc,data=d) # \alert{2: Xj on all other Xs}
%> coeftest(age_educ)
%t test of coefficients:
%            Estimate Std. Error t value Pr(>|t|)
%(Intercept) 49.88973    1.12154 44.4832   <2e-16 ***
%ppeduc      -0.39053    0.24153 -1.6169   0.1061
%
%> coeftest(lm(resid(y_educ)~resid(age_educ)))
%t test of coefficients:                  # \alert{3: Res from 1 on Res from 2}
%                   Estimate  Std. Error t value Pr(>|t|)
%(Intercept)      3.3832e-17  3.0545e-02   0.000  1.00000
%resid(age_educ) -4.4421e-03  1.8296e-03  -2.428  0.01529 *
%\end{Verbatim}
%
%\end{frame}
%
%\begin{frame}[fragile]
%\frametitle{Added Variable Plot for Age}
%\small
%Aside: We can reproduce the coefficient on age from the multivariate regression,
%as we saw before:
%\begin{Verbatim}[fontsize=\footnotesize, frame=single, label=R Code, commandchars=\\\{\}]
%> coeftest(lm(imgpro5mod1~ppeduc+ppage,data=d))
%t test of coefficients:
%              Estimate Std. Error t value Pr(>|t|)
%(Intercept)  2.0807961  0.1225670 16.9768  < 2e-16 ***
%ppeduc       0.1639231  0.0176238  9.3012  < 2e-16 ***
%ppage       -0.0044421  0.0018301 -2.4272  0.01533 *
%\end{Verbatim}
%\pause
%We can then construct the AV plot:
%\begin{Verbatim}[fontsize=\footnotesize, frame=single, label=R Code, commandchars=\\\{\}]
%> plot(y=resid(y_educ),resid(age_educ),
%+              col=c("red","red"),
%+              ylab="Residuals: Y on eduction",
%+              xlab="Residuals: Age on Education")
%> loess <- loess(resid(y_educ)~resid(age_educ))
%> points(y=loess$fitted,loess$x,col="blue",pch=19)
%\end{Verbatim}
%\end{frame}
%
%\frame{
%\frametitle{Added Variable Plot for Age}
%
%\begin{figure}[ht] \centering
%        \scalebox{.35}{\includegraphics{AV1.pdf}}
%     \end{figure}
%
%}
%
%\subsection{Component-Plus-Residual (CR) Plots}
%
%\frame{
%\frametitle{Component-Plus-Residual (CR) Plots}
%\small
%CR plots are an refinement of AV plots.
%
%\begin{enumerate}
%  \item Compute residuals from the full regression of $Y$ on all $X$: $\hat{u}_i = Y_i - (\hat{\beta}_0 + \hat{\beta}_1 X_{i1} + ... + \hat{\beta}_j X_{ij})$ \medskip
%  \item Compute the linear component of the partial relationship between $Y$ and $X_j$ from the full model, $lc_i = \hat{\beta}_j X_{ij}$ and add it to the residuals $\hat{u}_i^{j}=\hat{u}_i + lc_i$\medskip
%  \item Plot this partial residual $\hat{u}_i^{j}$ against $X_j$
%  \begin{itemize}
%    \item \texttt{R}: \texttt{crPlots(model)}
%  %  Stata: \texttt{cprplot xj};
%  \end{itemize}
%\end{enumerate}\bigskip
%\pause
%Facts about CR Plots
%\begin{itemize}
%  \item LS fit to the CR-plot has the slope $\hat{\beta}_j$ by construction\pause
%  \item Compared to AV plots, they have the $X_j$ on the X-axis which may be more intuitive to determine non-linearities and potential transformations\pause
%  \item Add Loess fit to detect various violations such as nonlinearity, heteroscedasticity, or outliers.
%\end{itemize}
%
%}
%
%\begin{frame}[fragile]
%\frametitle{Component-Plus-Residual Plots}
%\begin{Verbatim}[fontsize=\footnotesize, frame=single, label=R Code, commandchars=\\\{\}]
%> crPlots(lm(imgpro5mod1~ppeduc+ppage,data=d))
%\end{Verbatim}
%\begin{figure}[ht] \centering
%        \scalebox{.325}{\includegraphics{CR1.pdf}}
%     \end{figure}
%\end{frame}
%
%
%\frame{
%\frametitle{Limitations of CR Plots}
%
%\begin{itemize}
%  \item AV plots and CR plots can only reveal partial relationships
%  \item Oftentimes, these two-dimensional displays fail to uncover structure in
%  a higher-dimensional problem \medskip\pause
%  \item We may detect an interaction between $X_1$ and $X_2$ in a 3D scatterplot that we could miss in two scatterplots of $Y$ on each $X$\medskip\pause
%\item Cook (1993) shows that CR plots only work when:\pause
%\begin{enumerate}
%\item The relationship between $Y$ and $X_j$ is linear \medskip
%\item Other explanatory variables $(X_1,...,X_{j-1})$ are linearly related to $X_j$ \medskip
%\item Suggests that linearizing the relationship between the $X$s through transformations can be helpful\medskip
%\item Experience suggests weak non-linearities among $X$s do not invalidate CR plots
%\end{enumerate}
%\end{itemize}
%
%}
%
%
%\frame{
%\frametitle{How should we deal with nonlinearity?}
%\small
%Given we have a linear regression model, our options are somewhat limited. However
%we can partially address nonlinearity by:
%\begin{itemize}
%  \item Breaking categorical or continuous variables into dummy variables (e.g. education levels)\medskip
%  \item Including interactions\medskip
%  \item Including polynomial terms\medskip
%  \item Transformations such as logs
%  \item Generalized Additive Models (GAM)
%  \item Many more flexible, nonlinear regression models exist beyond the scope of this course (350C, 350D). 
%\end{itemize}
%}

 \begin{frame}[fragile]
\frametitle{Generalized Additive Models (GAM)}

Take the linear model,
$$
y_i = \beta_0 + x_{1i}\beta_1 + x_{2i}\beta_2 + x_{3i}\beta_3 + u_i
$$

Ideally one would have a model that learns 
$$
y_i = f( x_{1i}, x_{2i},x_{3i}) + u_i
$$
but this is hard.\\\bigskip
For GAMs, we maintain additivity, but instead of imposing linearity we allow flexible functional forms for each explanatory variable$$
y_i = \beta_0 + \alert{s_1(x_{1i})} + \alert{s_2(x_{2i})} + \alert{s_3(x_{3i})} + u_i
$$

 where $\alert{s_1(\cdot)}, \alert{s_2(\cdot)},\,\, \mbox{and} \,\,\alert{s_3(\cdot)}$ are
smooth functions that are estimated from the data.


\end{frame}

% \begin{frame}[fragile]
%\frametitle{Generalized Additive Models (GAM)}
%\small
%$$
%y_i = \beta_0 + \alert{s_1(x_{1i})} + \alert{s_2(x_{2i})} + \alert{s_3(x_{3i})} + u_i
%$$
%\begin{itemize}
%  \item GAMS are semi-parametric, they strike a compromise between nonparametric methods and parametric regression 
%  \item \alert{$s_j(\cdot)$} are usually estimated with locally weighted regression smoothers or cubic smoothing splines (but many approaches are possible)
%  \item They do NOT give you a set of regression parameters $\hat{\beta}$. Instead one obtains a graphical summary of how $E[Y|X_,X_2,...,X_k]$ varies with $X_1$ (estimates of $s_j(\cdot)$ at every value of $X_{i,j}$)
%  \item Theory and estimation are somewhat involved, but they are easy to use
%%  \begin{itemize}
%%    \item \texttt{gam.out     <- gam(y$\sim$s(x1)+s(x2)+x3)}\\
%%          \texttt{plot(gam.out)}\smallskip
%%    \item Good function:  \texttt{mgcv(gam)} % \texttt{vgam(vgam)}
%%  \end{itemize}
%\end{itemize}
%\end{frame}






 \begin{frame}[fragile]
\frametitle{Generalized Additive Models (GAM)}
\small
$$
y_i = \beta_0 + \alert{s_1(x_{1i})} + \alert{s_2(x_{2i})} + \alert{s_3(x_{3i})} + u_i
$$
\begin{itemize}
  \item GAMs are semi-parametric, they strike a compromise between nonparametric methods and parametric regression 
  \item GAMs are additive: create estimates of the regression surface by a sum of one-dimensional functions
  \item GAMs do not give you a set of regression parameters $\hat{\beta}$, but estimated functions $s_j(\cdot)$ 
    \item The functions can be summarized using a graphical summary of how $E[Y|X_,X_2,...,X_k]$ varies with $X_1$ (estimates of $s_j(\cdot)$ at every value of $X_{i,j}$)
  \item $s_j(\cdot)$ are usually estimated via back-fitting using locally weighted regression smoothers or various forms of splines (but many approaches are possible)
 \item Theory and estimation are somewhat involved, but they are easy to use
%  \begin{itemize}
%    \item \texttt{gam.out     <- gam(y$\sim$s(x1)+s(x2)+x3)}\\
%          \texttt{plot(gam.out)}\smallskip
%    \item Good function:  \texttt{mgcv(gam)} % \texttt{vgam(vgam)}
%  \end{itemize}
\end{itemize}
\end{frame}


 \begin{frame}[fragile]
\frametitle{Generalized Additive Models (GAM)}
\small
\begin{itemize}
  \item Assumption that the contribution of each covariate is additive is analogous to the assumption in linear regression that each component is estimated separately
\item In OLS
$$
y_i = \beta_0 + \sum_{j=1}^J  \beta_j   x_{ji} + u_i
$$  
  
  \item GAMs assume similar additivity, but  $y_i$ is modeled as sum of arbitrary functions of the $x$-s
  $$
  y_i = \beta_0 + \sum_{j=1}^J   s_j(x_{ij}) + u_i
  $$
errors are  assumed to be mean zero and constant variance
%  \begin{itemize}
%    \item \texttt{gam.out     <- gam(y$\sim$s(x1)+s(x2)+x3)}\\
%          \texttt{plot(gam.out)}\smallskip
%    \item Good function:  \texttt{mgcv(gam)} % \texttt{vgam(vgam)}
%  \end{itemize}
\end{itemize}
\end{frame}


\subsection{Fitting}


 \begin{frame}[fragile]
\frametitle{Generalized Additive Models (GAM)}
\small
\begin{itemize}
  \item How do we learn the arbitrary functions $s_j(x_{j}) $?
  \item Consider the case where the $X$s are independent, then we could simply estimate $s_j(x_{j}) $ by estimating a smooth of $Y$ on each of the $X$s separately
  \item Similarly in linear regression when the X-s are uncorrelated, the partial regression slopes are identical to the marginal regression slopes
  \item Since the $X$s are usually correlated, we need to remove the effects of other predictors
  \item  Solution: Backfitting algorithm that finds each function while controlling for the effects of the other $X$s
\end{itemize}
\end{frame}


 \begin{frame}[fragile]
\frametitle{Backfitting}
\small
\begin{itemize}
  \item Suppose that we have a two predictor additive model:$$
y_i = \beta_0 + \alert{s_1(x_{1i})} + \alert{s_2(x_{2i})}  + u_i
$$
\item If we knew the partial regression function $s_2$ but not $s_1$ we could re-arrange
$$
y_i - \alert{s_2(x_{2i})} = \beta_0 + \alert{s_1(x_{1i})} +   + u_i
$$ and use this to estimate $s_1$ by smoothing $y_i - \alert{s_2(x_{2i})} $ against $x_1$.
\item Knowing one partial function allows us to find the other
\item We don't know $s_2$, but we can approximate it with an initial guess and then proceed iteratively $\rightarrow$ backfitting!
\end{itemize}
\end{frame}


 \begin{frame}[fragile]
\frametitle{Backfitting}
\small
\begin{enumerate}
  \item Express variables in mean-deviation form to eliminate intercepts (partial regressions sum to zero)
  \item Run simple linear regression to get initial estimates of each partial regression function$$
y_i -\bar{y} =  \beta_1 ( x_{1i} - \bar{x}_1 ) +  \beta_2 ( x_{2i} - \bar{x}_2 )  + \epsilon_i
  $$
  $$
  y_i^\star =  \beta_1 x_{1i}^\star  +  \beta_2 x_{2i}^\star   + \epsilon_i
  $$
 \item Use estimates as step 0 in iterative estimation procedure 
 
 $$
 \hat{s_1}^{(0)} =  \hat{\beta}_1 x_{1i}^\star \,\, \mbox{and}  \,\,   \hat{s_2}^{(0)} =  \hat{\beta}_2 x_{2i}^\star 
 $$
 \item Then we find partial residuals for $x_1$ which removes $y$ from its linear relationship with $x_2$, but retains relationship between $y$ and $x_1$. Partial residuals for $x_1$ are$$
 \epsilon_{i[1]}^{(1)}  =  y_i^\star -  \hat{\beta}_2 x_{2i}^\star = \epsilon_i + \hat{\beta}_1 x_{1i}^\star
 $$
 and we do the same for $x_2$
\end{enumerate}
\end{frame}

 \begin{frame}[fragile]
\frametitle{Backfitting}
\small
\begin{enumerate}
\setcounter{enumi}{4}
  \item Now we can smooth the partial residuals against their respective $X$s to provide an updated estimate of $s()$. $$
  \hat{s_k}^{(1)} =  \mbox{smooth} [  \epsilon_{i[k]}^{(1)} \,\, \mbox{on}   \,\, x_{ik} ] 
  $$ where any reasonable smoother could be used (e.g. a loess or spline)
  \item Iterate in finding updated estimates of the functions until the partial functions converge, ie. they no longer change from one iteration to the next
  \end{enumerate}
Key result: We now have estimates of $s_k(x_{ij})$ for every value of $x_j$ and we have reduced the multiple regression problem into a series of two-dimensional partial regression functions.\\\bigskip This makes interpretation easy: we can plot functions which show the partial effects of each $x$ on $y$ (controlling for other $X$s)
\end{frame}

\subsection{Interpretation}

 \begin{frame}[fragile]
\frametitle{Interpretation: GAM Partial Regression Plots}
\small
\begin{itemize}
  \item  A plot of of $x_j$ versus $s_j(x_j)$ shows the relationship between $x_j$ and $y$ holding constant the other variables in the model
  \item Since $y$ is in mean deviation form, the smooth term $s_j(x_j)$ is also centered and thus each plot represents how $y$ changes \alert{relative to its mean} with changes in $x$
  \item Interpretation of scale of GAM plots is also straightforward
  
  \begin{itemize}
  
 \item Value of 0 on Y-axis is the mean of $y$\medskip
 \item Deviations from zero are deviations from the mean and we can add the mean to obtain the fitted values 
      \end{itemize}
  
\end{itemize}
\end{frame}

\frame{
\frametitle{Which Smoother to Use? }

\begin{figure}[ht] \centering
        \scalebox{1.2}{\includegraphics{splines.pdf}}
     \end{figure}
  Need to find the right balance between under and over-smoothing   
}


 \begin{frame}[fragile]
\frametitle{Generalized Additive Models (GAM)}

The GAM approach can be extended to allow \alert{interactions} ($s_{12}(\cdot)$) between explanatory variables, but this eats up degrees of freedom so you need a lot of data (often thin plate splines are used here).

$$
y_i = \beta_0 + \alert{s_{12}(x_{1i}, x_{2i})} + \alert{s_3(x_{3i})} + u_i
$$\bigskip \pause

It can also be used for \alert{hybrid models} where we model some variables as parametrically and other with a flexible function:

$$
y_i = \beta_0 + \beta_1 x_{1i} + \alert{s_2(x_{2i})} + \alert{s_3(x_{3i})} + u_i
$$

\end{frame}

\subsection{Examples}

\frame{
\frametitle{Effect of Disaster Aid on Vote Share}

\begin{figure}[ht] \centering
        \scalebox{.7}{\includegraphics{gamelbe1.pdf}}
     \end{figure}
     
     \scriptsize{Bechtel and Hainmueller. 2011. ``How Lasting is Voter Gratitude? An Analysis of the Short- and Long-term Electoral Returns to Beneficial Policy". American Journal of Political Science.}

}

\frame{
\frametitle{Effect of Disaster Aid on Vote Share}
\vspace{-.3in}
\begin{figure}[ht] \centering
        \scalebox{.4}{\includegraphics[angle=270]{elbe.pdf}}
     \end{figure}
     
 %    \scriptsize{Bwechtel and Hainmueller. 2011. ``How Lasting is Voter Gratitude? An Analysis of the Short- and Long-term Electoral Returns to Beneficial Policy". American Journal of Political Science.}

}

\frame{
\frametitle{GAM Fit to Diadic Democracy and Militarized Disputes}

\begin{figure}[ht] \centering
        \scalebox{.6}{\includegraphics{dempeace.pdf}}
     \end{figure}

}

\subsection{Implementing GAMs in R}


 \begin{frame}[fragile]
\frametitle{Generalized Additive Models (GAM)}

\begin{itemize}
\item Workhorse function is \textit{gam()} in \textit{mgcv} package \medskip
\item The formula takes the same form as the \textit{glm} function except now we have the option of having parametric terms and smoothed estimates \medskip
\item Smooths will be fit to any variable specified with the \textit{s(x)} argument (can also use other smoothers \textit{te, ti or t2}) \medskip
\item Example: \textit{gam.out $<$- gam(y$\sim$x1+s(x2)+s(x3),data=data)}\\ (add \textit{family} to use with any standard GLM) \medskip
\item Can use \textit{plot(gam.out)}, \textit{predict(gam.out)}, and \textit{summary(gam.out)}  methods to interpret results
\item The summary function returns tests for each smooth, the degrees of freedom for each smooth, and an adjusted R- square for the model. 
\item Statistical inference and hypothesis testing is based on the deviance with approximate empirical degrees of freedom
\end{itemize}

\end{frame}


\frame{
\frametitle{Example: Attitudes Towards Immigration}

\begin{itemize}
  \item Outcome: Pro-Immigration Attitudes
\begin{footnotesize}
  \begin{itemize}
    \item 1 Strongly opposed to increase in immigration
    \item .
    \item 5 Strongly in favour of increase in immigration
  \end{itemize}
\end{footnotesize}
  \item Highest Educational Attainment:\\
\begin{footnotesize}
\begin{itemize}
  \item 1 Less than high school
  \item 2 Some high school, no diploma
  \item 3 Graduated from high school- Diploma or equivalent (GED)
  \item 4 Some college, no degree
  \item 5 Associate degree (AA, AS)
  \item 6 Bachelor's degree
  \item 7 Master's degree
  \item 8 Professional degree (MD, DDS, LLB, JD)
  \item 9 Doctorate degree
\end{itemize}
\end{footnotesize}
%  \item Household Income
%\begin{footnotesize}
%  \begin{itemize}
%    \item 1  Less than \$5,000
%    \item
%    \item 19 More than \$175,000
%\end{itemize}
%\end{footnotesize}
  \item Age of Respondent
\begin{footnotesize}
  \begin{itemize}
    \item 18 years to 93 years
\end{itemize}
\end{footnotesize}

\end{itemize}

}
%
\frame{
\frametitle{Attitudes Towards Immigration (2008)}

\begin{figure}[ht] \centering
        \scalebox{.55}{\includegraphics{fig0-nl.pdf}}
     \end{figure}

}

\frame{
\frametitle{Immigration Attitudes and Education}

\begin{figure}[ht] \centering
        \scalebox{.45}{\includegraphics{fig1i-nl.pdf}}
     \end{figure}

}

\frame{
\frametitle{Immigration Attitudes and Education}

\begin{figure}[ht] \centering
        \scalebox{.45}{\includegraphics{fig2i-nl.pdf}}
     \end{figure}

}


\begin{frame}[fragile]
\frametitle{Immigration Attitudes and Education}

\begin{verbatim}[fontsize=\footnotesize, frame=single, label=R Code]
> mod0 <- lm(imgpro5mod1~ppeduc, data=d)
> coeftest(mod0)

t test of coefficients:

            Estimate Std. Error t value  Pr(>|t|)
(Intercept) 1.859180   0.081894 22.7022 < 2.2e-16 ***
ppeduc      0.165658   0.017636  9.3929 < 2.2e-16 ***
---
Signif. codes:  0 ‘***’ 0.001 ‘**’ 0.01 ‘*’ 0.05 ‘.’ 0.1 ‘ ’ 1
\end{verbatim}

\end{frame}

\begin{frame}[fragile]
\frametitle{Immigration Attitudes and Education}

\begin{verbatim}[fontsize=\footnotesize, frame=single, label=R Code]
> mod0a <- lm(imgpro5mod1~factor(ppeduc), data=d)
> coeftest(mod0a)

t test of coefficients:

                Estimate Std. Error t value  Pr(>|t|)
(Intercept)     2.173913   0.253720  8.5682 < 2.2e-16 ***
factor(ppeduc)2 0.261461   0.272848  0.9583 0.3380747
factor(ppeduc)3 0.096465   0.259456  0.3718 0.7100945
factor(ppeduc)4 0.324500   0.262820  1.2347 0.2171310
factor(ppeduc)5 0.461008   0.275907  1.6709 0.0949432 .
factor(ppeduc)6 0.673605   0.263864  2.5528 0.0107779 *
factor(ppeduc)7 1.059170   0.274784  3.8546 0.0001206 ***
factor(ppeduc)8 0.740373   0.326614  2.2668 0.0235363 *
factor(ppeduc)9 1.186087   0.351565  3.3737 0.0007595 ***
\end{verbatim}

\end{frame}


\frame{
\frametitle{Immigration Attitudes and Age}

\begin{figure}[ht] \centering
        \scalebox{.45}{\includegraphics{fig5i-nl.pdf}}
     \end{figure}

}

\frame{
\frametitle{Immigration Attitudes and Age}

\begin{figure}[ht] \centering
        \scalebox{.45}{\includegraphics{fig6i-nl.pdf}}
     \end{figure}

}

\begin{frame}[fragile]
\frametitle{Immigration Attitudes and Age}

\begin{verbatim}[fontsize=\footnotesize, frame=single, label=R Code]
> mod1 <- lm(imgpro5mod1~ppage, data=d)
> coeftest(mod1)
t test of coefficients:
              Estimate Std. Error t value  Pr(>|t|)
(Intercept)  2.8201106  0.0957797  29.444 < 2.2e-16 ***
ppage       -0.0051324  0.0018772  -2.734  0.006326 **
>
> mod1a <- lm(imgpro5mod1~ppage+I(ppage^2), data=d)
> coeftest(mod1a)
t test of coefficients:
               Estimate  Std. Error t value  Pr(>|t|)
(Intercept)  3.60493245  0.23929787 15.0646 < 2.2e-16 ***
ppage       -0.04123139  0.01026562 -4.0165 6.184e-05 ***
I(ppage^2)   0.00036702  0.00010262  3.5763 0.0003589 ***
\end{verbatim}

\end{frame}

\frame{
\frametitle{Immigration Attitudes, Education and Age}

\begin{figure}[ht] \centering
        \scalebox{.45}{\includegraphics{fig3d1-nl.pdf}}
     \end{figure}

}

\frame{
\frametitle{Immigration Attitudes, Education and Age}

\begin{figure}[ht] \centering
        \scalebox{.45}{\includegraphics{fig3d2-nl.pdf}}
     \end{figure}
}


\frame{
\frametitle{Immigration Attitudes, Education and Age}

Go through GAM Code

}
%
%
%\frame{
%\frametitle{GAM Fit to Attitudes Toward Immigration}
%
%\begin{figure}[ht] \centering
%        \scalebox{.4}{\includegraphics{gam1.pdf}}
%     \end{figure}
%
%}
%
%\frame{
%\frametitle{GAM Fit to Attitudes Toward Immigration}
%
%\begin{figure}[ht] \centering
%        \scalebox{.4}{\includegraphics{gam2.pdf}}
%     \end{figure}
%
%}
%
%\frame{
%\frametitle{GAM Fit to Attitudes Toward Immigration}
%
%\begin{figure}[ht] \centering
%        \scalebox{.4}{\includegraphics{gam3.pdf}}
%     \end{figure}
%
%}
%
%\frame{
%\frametitle{GAM Fit to Attitudes Toward Immigration}
%
%\begin{figure}[ht] \centering
%        \scalebox{.4}{\includegraphics{gam3a.pdf}}
%     \end{figure}
%
%}
%
%\frame{
%\frametitle{GAM Fit to Attitudes Toward Immigration}
%
%\begin{figure}[ht] \centering
%        \scalebox{.4}{\includegraphics{gam4.pdf}}
%     \end{figure}
%
%}
%
%\frame{
%\frametitle{GAM Fit to Attitudes Toward Immigration}
%
%\begin{figure}[ht] \centering
%        \scalebox{.4}{\includegraphics{gam4a.pdf}}
%     \end{figure}
%
%}


\frame{
\frametitle{Further Readings on More Flexible Regression Methods}
\small
\begin{itemize}
      \item Beck, N. and Jackman, S. 1998. Beyond Linearity by Default: Generalized Additive Models. \emph{American Journal of Political Science}. 
  \item Wood (2003). ``Thin plate regression splines.'' \emph{Journal of the Royal Statistical Society: Series B}.\medskip
      \item Hastie, T.J. and Tibshirani, R.J. 1990. \emph{General Additive Models}.\medskip
%    \item Hastie, Tibshirani, and Friedman (2009). \emph{The Elements of Statistical Learning: Data Mining, Inference, and Prediction}. Springer. \medskip
\end{itemize}

}

%\section{Summary: Regression Diagnostics}
%
%
% \begin{frame}[fragile]
%\frametitle{Diagnostics of Violations of Regression Assumptions}
%\vspace{-.15in}
%\begin{figure}[ht] \centering
%        \scalebox{.45}{\includegraphics{summaryofdiagnostics2.pdf}}
%     \end{figure}
%
%\end{frame}


\end{document}
