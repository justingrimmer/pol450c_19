\documentclass{beamer}

%\usepackage[table]{xcolor}
\mode<presentation> {
  \usetheme{Boadilla}
%  \usetheme{Pittsburgh}
%\usefonttheme[2]{sans}
\renewcommand{\familydefault}{cmss}
%\usepackage{lmodern}
%\usepackage[T1]{fontenc}
%\usepackage{palatino}
%\usepackage{cmbright}
  \setbeamercovered{transparent}
\useinnertheme{rectangles}
}
%\usepackage{normalem}{ulem}
%\usepackage{colortbl, textcomp}
\setbeamercolor{normal text}{fg=black}
\setbeamercolor{structure}{fg= black}
\definecolor{trial}{cmyk}{1,0,0, 0}
\definecolor{trial2}{cmyk}{0.00,0,1, 0}
\definecolor{darkgreen}{rgb}{0,.4, 0.1}
\newtheorem{prop}{Proposition}
\newtheorem{defn}{Definition}
\newtheorem{cor}{Corollary}
\newtheorem{lem}{Lemma}

\newcommand\indep{\protect\mathpalette{\protect\independenT}{\perp}}
\def\independenT#1#2{\mathrel{\rlap{$#1#2$}\mkern2mu{#1#2}}}

\usefonttheme{professionalfonts} % using non standard fonts for beamer
\usefonttheme{serif}

\beamertemplatesolidbackgroundcolor{white}  \setbeamercolor{alerted
text}{fg=red}
\usepackage{multirow}

\setbeamertemplate{caption}[numbered]\newcounter{mylastframe}

%\begin{beamerboxesrounded}[upper=uppercol,lower=lowercol,shadow=true]{Block}
%$A = B$.
%\end{beamerboxesrounded}}
\renewcommand{\familydefault}{cmss}
\usepackage[all]{xy}
\usepackage{tikz}
\usetikzlibrary{arrows}
%\usepackage{palatino}
%\usepackage{eulervm}
\usecolortheme{lily}
%\usepackage[latin1]{inputenc}

\title[Methodology III] % (optional, nur bei langen Titeln nötig)
{Political Methodology III: Model Based Inference}

\author{Justin Grimmer}
\institute[Stanford University]{Professor\\Department of Political Science \\  Stanford University}
\vspace{0.3in}

\date{May 7th, 2019}

\begin{document}
\begin{frame}
\titlepage
\end{frame}


\begin{frame}
\frametitle{Model Based Inference}

\begin{itemize}
\item[1)] Likelihood inference
\item[2)] Logit/Probit
\item[3)] Ordered Probit
\item[4)] Choice Models:
\item[5)] Count Models
\item[6)] Survival Models
\item[7)] \alert{Hypothesis Tests + Model Checking in Likelihood}
\begin{itemize}
\item[-] Likelihood Ratios, Wald, and Score tests
\item[-] Model Checking: analysis of residuals, hat values, etc.
\end{itemize}
\end{itemize}

\end{frame}


\begin{frame}
\frametitle{Simple Example: Antiobama Speech}

We'll  use the speech data from the problem set, as follows:

\begin{itemize}
\item[-] $Y_{i} = 1 $ if representative says {\tt obamacare} or {\tt big government} during the year, 0 otherwise
\item[-] $\boldsymbol{X}_{i} = (1, I(\text{Year} = 2010)_{i}, \text{Democrat}_{i}, \text{DW-Nom}_{i}) $
\end{itemize}

\begin{eqnarray}
Y_{i}  &  \sim &   \text{Bernoulli}(\pi_{i}) \nonumber \\
\pi_{i}     & = & \text{logit}^{-1}(\boldsymbol{X}_{i}^{'} \boldsymbol{\beta})  = \frac{1}{1 + \exp(-\boldsymbol{X}_{i}^{'} \boldsymbol{\beta}) } \nonumber
\end{eqnarray}


Which covariates do we include? $\leadsto$ depends on goal.  \\
\begin{itemize}
\item[-] Predictive goal $\leadsto$ replicate task
\item[-] Model fitting $\leadsto$ do covariates increase likelihood? Can we drop them?
\end{itemize}

\end{frame}


\begin{frame}
\frametitle{Hypothesis Testing --- Likelihood Ratio Test}
\begin{itemize}[<+->]
\item Null ($H_0$): $h_1(\boldsymbol{\beta}) = \cdots = h_Q(\boldsymbol{\beta}) = 0$ ($Q$ equality constraints)
\item Alternative ($H_1$): No such constraints
\medskip
\item Let $\widehat{\boldsymbol{\beta}}_{R} = \widehat{\boldsymbol{\beta}}_{MLE\mid H_0}$ (restricted MLE) and
$\widehat{\boldsymbol{\beta}}_{UR} = \widehat{\boldsymbol{\beta}}_{MLE}$ (original MLE)
\item \alert{Likelihood ratio} (LR) test: If $H_0$ is true, $L(\widehat{\boldsymbol{\beta}}_R)$
should be equal to $L(\widehat{\boldsymbol{\beta}}_{UR})$ except for sampling variability
\item LR statistic:
    $$ LR(Y) \ \equiv \ -2\log\frac{L(\widehat{\boldsymbol{\beta}}_{R})}{L(\widehat{\boldsymbol{\beta}}_{UR})}
     \ = \ 2\left[\ell(\widehat{\boldsymbol{\beta}}_{UR}) - \ell(\widehat{\boldsymbol{\beta}}_{R})\right] $$
\item We can show that $LR(Y) \stackrel{d}{\longrightarrow} \chi_Q^2$
\item Works for testing any nested models \\
    --- model under $H_0$ has to be a special case of model under $H_1$
\end{itemize}

\end{frame}

\begin{frame}

\small
\begin{semiverbatim}

un\_rest\_reg<- glm(once$\sim$two\_10 + dem + dw\_nom,

\hspace{0.15in} data = speech\_dat, family = binomial(link = logit))


rest\_reg<- glm(once$\sim$1, family= binomial(link = logit))


\#\#calculating the likelihood ratio

log\_lik<- function(pars, X, Y)\{

  \hspace{0.15in} y.tilde<- X\%*\%pars

  \hspace{0.15in} probs<- plogis(y.tilde)

  \hspace{0.15in} log\_out<- Y\%*\%log(probs) + (1-Y)\%*\%log(1 - probs)

  \hspace{0.15in} return(log\_out)

  \}

 X<- cbind(1, two\_10, dem, speech\_dat\$dw\_nom)


 un\_rest<- log\_lik(un\_rest\_reg\$coef, X, once)


 rest<- log\_lik(rest\_reg\$coef, as.matrix(rep(1, nrow(X))), once)


 > 2 * un\_rest - 2*rest

\hspace{0.15in}        [,1]

[1,] 433.996








 \#\#get the same statistic automatically from glm

 diff<- un\_rest\_reg\$null.deviance - un\_rest\_reg\$deviance

 > diff
[1] 433.996

 1 - pchisq(diff, 3) \#\#very small!

[1] 0

\end{semiverbatim}



\end{frame}

\begin{frame}

\begin{semiverbatim}
 > 2 * un\_rest - 2*rest

\hspace{0.15in}        [,1]

[1,] 433.996

 \#\#get the same statistic automatically from glm

 diff<- un\_rest\_reg\$null.deviance - un\_rest\_reg\$deviance

> diff

[1] 433.996


 1 - pchisq(diff, 3) \#\#very small!

[1] 0

\end{semiverbatim}


\end{frame}


\begin{frame}
\frametitle{Hypothesis Testing --- Wald Test}
\begin{itemize}[<+->]
\item \alert{Wald test}: If true, the null $h_1(\boldsymbol{\beta})=\cdots =h_Q(\boldsymbol{\beta})=0$
should approximately hold even if we substitute $\widehat{\boldsymbol{\beta}}_{UR}$ for $\boldsymbol{\beta}$.  Call $\boldsymbol{h}(\boldsymbol{\beta}) = \left(h_{1} (\boldsymbol{\beta}), \hdots, h_{Q}(\boldsymbol{\beta}) \right)$
\item Wald statistic: Use asymptotic distribution of $\widehat{\boldsymbol{\beta}}$ and representation of restrictions, properties of normal distribution to obtain form
{\small
    $$ W \ \equiv \ h(\widehat{\boldsymbol{\beta}}_{UR})^{'} \left[\left(\frac{\partial h(\boldsymbol{\beta})}
    {\partial \boldsymbol{\beta}}\Bigl|_{\boldsymbol{\beta}=\widehat{\boldsymbol{\beta}}_{UR}} \right)^{'}
    \widehat{\text{Var}(\widehat{\boldsymbol{\beta}}_{UR})}
    \left(\frac{\partial h(\boldsymbol{\beta})}{\partial\boldsymbol{\beta}}\Bigl|_{\boldsymbol{\beta}=\widehat{\boldsymbol{\beta}}_{UR}}\right)\right]^{-1}
    h(\widehat{\boldsymbol{\beta}}_{UR})$$
}

\item The ``meat'' $\simeq \widehat{\text{Var}(h(\widehat{\boldsymbol{\beta}}_{UR}))}$
(\alert{Delta method} )
\item Choose any $\widehat{\text{Var}(\widehat{\boldsymbol{\beta}}_{UR})}$ as appropriate (e.g. Huber-White)
\item We can show that $W \stackrel{d}{\longrightarrow} \chi_Q^2$
\bigskip
\item An important special case: $Q=1$ and $H_0: \beta = 0$
\item In this case, we can use the \alert{$z$ statistic}:
{\small
$$ z \ = \ W^{1/2} \ = \ \frac{\widehat{\boldsymbol{\beta}}_{UR}}{\text{s.e.}(\widehat{\boldsymbol{\beta}}_{UR})}
 \ \stackrel{d}{\longrightarrow} \ \text{N}(0,1)$$
}
%where $s.e.(\widehat{\beta}_{UR})$ is the sqrt of the diagonal element of $\widehat{\text{Var}(\widehat{\beta}_{UR})}$
\end{itemize}

\end{frame}


\begin{frame}

\small
\begin{semiverbatim}

> un\_rest\_reg\$coef\%*\%solve(vcov(un\_rest\_reg))\%*\%un\_rest\_reg\$coef

         [,1]

[1,] 225.2437


> 1 - pchisq(225.2437, 3)

[1] 0


\end{semiverbatim}



\end{frame}



\begin{frame}
\frametitle{Hypothesis Testing --- Score Test}
\begin{itemize}[<+->]
\item At the unrestricted MLE $\hat\theta_{UR}$, $\sum_{i=1}^{N} s_{i} (\widehat{\boldsymbol{\beta}}_{UR})=s(\widehat{\boldsymbol{\beta}}) = 0$ by construction
\item \alert{Score test}: If the null is true, $s(\widehat{\boldsymbol{\beta}}_{R})$
  should also equal zero except for sampling variability
\item Score statistic: Use asymptotic distribution and properties of normal distribution to  ``standardize" $s(\widehat{\boldsymbol{\beta}}_{R})$
$$ LM \ = \ s(\widehat{\boldsymbol{\beta}}_{R})^{'} \widehat{\text{Var}(\widehat{\boldsymbol{\beta}}_{R})} s(\widehat{\boldsymbol{\beta}}_{R})
\ \stackrel{d}{\longrightarrow} \ \chi_Q^2$$

\item For $\widehat{\boldsymbol{\beta}}_{QMLE}$, the expression is more complicated
\item Also known as the \alert{Lagrange multiplier} (LM) test due to an alternative derivation

\end{itemize}

\end{frame}



\begin{frame}

\small
\begin{semiverbatim}

 score\_func<- function(coef, X, Y)\{

  \hspace{0.15in} y.tilde<- X\%*\%coef

  \hspace{0.15in} probs<- plogis(y.tilde)

 \hspace{0.15in}  out<- t(Y - probs)\%*\%X

 \hspace{0.15in}  return(out)
   \}



> round(score\_func(un\_rest\_reg\$coef, X, once), 2)


[1,] 0      0   0 0


rest\_score<- score\_func(c(rest\_reg\$coef, 0, 0, 0), X, once)


> round(rest\_score,2)

[1,] 0 -6.30 -128.92 129.51

\end{semiverbatim}
\end{frame}


\begin{frame}
\small
\begin{semiverbatim}

 hess\_func<- function(coef, X, Y)\{

   \hspace{0.15in}  y.tilde<- X\%*\%coef

  \hspace{0.15in}   probs<- plogis(y.tilde)

  \hspace{0.15in}   base<- matrix(0, nrow = len(coef), ncol = len(coef))

   \hspace{0.15in}  for(z in 1:nrow(X))\{

   \hspace{0.25in}    base<- base + probs[z]*(1 - probs[z])* X[z,]\%*\%t(X[z,])

     \hspace{0.15in}  \}

     \hspace{0.15in} return(base)

   \}


 rest\_hess<- solve(hess\_func(c(rest\_reg\$coef, 0, 0, 0), X, once))

>rest\_score\%*\%rest\_hess\%*\%t(rest\_score)

[1,] 395.0382

> 1- pchisq(395, 3)

[1] 0

\end{semiverbatim}

\end{frame}


\begin{frame}
\frametitle{Comparing The Three Tests}
\vspace{-.2in}
\begin{center}
\includegraphics[width=\textwidth]{threetests.jpg}
\end{center}
\vspace{-.2in}
\begin{itemize}
\item All asymptotically equivalent
\item But can be quite different in small samples
\end{itemize}
\pause
{\small
\begin{tabular}{cll}
\hline\hline
   & {\it Pros}                           & {\it Cons} \\
\hline\hline
LR & Most powerful (Neyman-Pearson) & Must compute both $\hat\theta_{UR}$ and $\hat\theta_{R}$ \\
   &                                & Cannot be easily robustified \pause \\
\hline
W  & Only need $\hat\theta_{UR}$     & Not invariant to transformation \\
   & Easily robustified by sandwich & (e.g. $\theta_1/\theta_2 = 1$ vs. $\theta_1=\theta_2$) \pause \\
\hline
LM & Only need $\hat\theta_{R}$      & $\hat\theta_{R}$ often difficult to estimate \\
\hline\hline
\end{tabular}
}
\end{frame}


\frame{

  \frametitle{Generalized Linear Models}
\small
  \begin{itemize}
  \item Many of the models we have learned so far all assume that $Y_i$ is a
    (stochastic) function of the \alert{linear predictor},
    $X_i^\top\beta$
  \item They also share many characteristics, e.g. the form of the
    score and Hessian functions
    \pause
  \item In fact, many are special cases of the
  \alert{generalized linear model (GLM)}
  \item Here, we provide a general treatment of GLMs to study those models
  more systematically
    \pause
\item 3 components of a GLM
    \pause
 \begin{enumerate}
  \item Systematic component: \ $X_i^\top\beta$
    \begin{itemize}
      \item Must be a linear function of $X_i$
    \end{itemize}
\medskip
    \pause
  \item Random component: \ $f(Y; \theta, \phi)$
      \begin{itemize}
      \item Must be in the \alert{exponential family}
      \item $\theta$ is called the canonical parameter
      \item $\phi$: is called the dispersion parameter

      \end{itemize}
\medskip
    \pause
 \item Link function:  \ $g(\mu_i) = X_i^\top\beta$ where $\mu_i = \text{E}(Y_i\mid
   X_i)$
      \begin{itemize}
        \item Must be monotonic and differentiable wrt $\mu_i$
      \end{itemize}
   \end{enumerate}

% \item Key properties:
%    \begin{itemize}
%      \item $\beta$ can be consistently estimated without fully
%        specifying $f(Y)$ (i.e., via QMLE), as long as $\mu_i$  and
%        $\V(Y_i)$ is correctly specified
%      \item Lots of the results for the classical linear model
%        (e.g. diagnostics) generalize to GLMs

%    \end{itemize}
 \end{itemize}


}


\begin{frame}
\frametitle{Exponential Family of Distributions}
Any distribution in the exponential family has the density of
    the following form:
$$f(y| \theta, \phi) =
\exp\left\{\frac{y\textcolor<3>{red}{\theta}-\textcolor<4>{blue}{b(}\textcolor<3>{red}{\theta}\textcolor<4>{blue}{)}}{a(\phi)}+c(y,\phi)\right\}$$

    \pause
Example: Poisson$(\lambda)$:
      $$ \Pr(Y_i=y\mid\lambda) \ = \ \frac{\exp(-\lambda) \lambda^{y}}{y!}\ = \ \exp\left\{y\textcolor<3>{red}{\log\lambda} - \textcolor<4>{blue}{\exp(}\textcolor<3>{red}{\log\lambda}\textcolor<4>{blue}{)} - \log y!  \right\}$$
    \pause\pause\pause
      $\Longrightarrow$ $\theta = \log\lambda$, $\phi=1$,
      $a(\phi)=\phi, b(\theta)=\exp(\theta)$, and $c=-\log y!$


\end{frame}


\begin{frame}


$$f(y_{i} | \theta_{i}, \phi) =
\exp\left\{\frac{y\textcolor<3>{red}{\theta_{i}}-\textcolor<4>{blue}{b(}\textcolor<3>{red}{\theta}\textcolor<4>{blue}{)}}{a(\phi)}+c(y,\phi)\right\}$$


Example: Bernoulli$(\pi_{i}) $:

\begin{eqnarray}
\Pr(Y_{i} = y_{i} | \pi_{i}) & = &  \pi_{i}^{y_{i} } (1- \pi_{i})^{1 - y_{i}} \nonumber \\
                             & = & \exp\left[ \frac{y_{i} \textcolor<3>{red}{\theta_{i}} - \textcolor<4>{blue}{\log[1 + \exp(}\textcolor<3>{red}{\theta_{i}} \textcolor<4>{blue}{)]}}{1} - 0  \right] \nonumber
\end{eqnarray}
where:
\begin{eqnarray}
\textcolor<3>{red}{\theta_{i} }& = & \text{logit}(\pi_{i} ) \ = \ \log\left(\frac{\pi_{i} }{1 - \pi_{i}}\right) \nonumber \\
\textcolor<4>{blue}{b(} \textcolor<3>{red}{\theta_{i}} \textcolor<4>{blue}{)} & = & \textcolor<4>{blue}{\log [ 1 + \exp(}\textcolor<3>{red}{\theta_{i}} \textcolor<4>{blue}{)]} \nonumber \\
a(\phi) & = & 1 ; c(y_{i}, \phi) = 0 \nonumber
\end{eqnarray}




\end{frame}



\frame{
  \frametitle{Properties of the Exponential Family}
  \begin{itemize}
  \item Mean is a function of $\theta$ and given by
    $$\text{E}(Y) \ \equiv \ \mu \ = \ b^\prime(\theta)$$
    \pause
    \vspace{-.15in}
  \item Variance is a function of $\theta$ and $\phi$ and given by
    $$\text{Var}(Y) \ \equiv \ V \ = \ b^{\prime\prime}(\theta)a(\phi)$$

    \pause
\smallskip
  \item Common forms of $a(\phi)$: $1$ (Poisson, Bernoulli), $\phi$ (normal, Gamma), and $\phi/\omega_i$ (binomial)    \pause
\medskip
  \item $b^{\prime\prime}(\theta)$ is called the \alert{variance function}
    \pause
\medskip
 \end{itemize}

}



\begin{frame}

\begin{itemize}
  \item In the Poisson model, $\theta_i = \log\lambda_i$, $a(\phi)=1$ and
    $b(\theta_i)=\exp(\theta_i)$ \\
    \pause \smallskip
    $\Rightarrow$ $\text{E}(Y_i) = \frac{\partial b(\theta_i)}{\partial \theta_i} = \exp(\theta_i) = \lambda_i$
    and $\text{Var}(Y_i) = \frac{\partial^2b(\theta_i)}{\partial \theta_i^2} = \exp(\theta_i) = \lambda_i$
\item In the Bernoulli model, $\theta_{i}  = \text{logit}(\pi_{i})$, $a(\phi) = 1$ and $b(\theta_{i}) = \log[1 + \exp(\theta_{i})] $.  \\
\pause \smallskip
$\Rightarrow$   $\text{E}(Y_i) = \frac{\partial b(\theta_i)}{\partial \theta_i} \ =\ \frac{\exp(\theta_{i})}{1 + \exp(\theta_{i})} \ = \ \text{logit}^{-1}(\theta_{i} )\ = \  \text{logit}^{-1}(\text{logit}(\pi_{i}))  = \pi_{i} $ \\
$\text{Var}(Y_{i}) = \frac{\partial^2b(\theta_i)}{\partial \theta_i^2} = \frac{\exp(\theta_{i})(1 + \exp(\theta_{i}))  - \exp(\theta_{i}) \exp(\theta_{i})}{ (1 + \exp(\theta_{i}))^{2} } = \frac{\exp(\theta_{i})}{1 + \exp(\theta_{i})} \frac{1}{1 + \exp(\theta_{i})}\ = \ \pi_{i} (1- \pi_{i})  $


\end{itemize}



\end{frame}


\frame{

  \frametitle{Link Functions}

  \begin{itemize}
  \item \alert{Link function}: $g(\mu_i) = X_i^\top\beta$
  \item Defines the relationship between $X_i^\top\beta$ and the mean $\mu_i$
  \item Must map the real line onto the possible range of $\mu_i$

\medskip
    \pause
  \item Recall that $\mu_i = b'(\theta_i)$
  \item Therefore, $\theta_i$ is always a (often simple) function of $X_i^\top\beta$
  \item In particular, when $\theta_i = X_i^\top\beta$, the link is called the
    \alert{canonical link}
  \smallskip
  \pause
     \begin{itemize}
     \item In Poisson, $\theta_i = \log(\lambda_i) = \pause \log(\exp(X_i^{'}\beta)) = X_i^{'}\beta$ \\ \pause
     \smallskip
     $\longrightarrow$ $\exp^{-1} = \log$ is the canonical link function\pause
     \item In Bernoulli $\theta_{i} = \text{logit}(\pi_{i}); \pi_{i} = \frac{\exp(\boldsymbol{X}_{i}^{'} \boldsymbol{\beta})}{ 1+ \exp(\boldsymbol{X}_{i}^{'} \boldsymbol{\beta})} $ \\ \pause
     \smallskip
     $\theta_{i} = \text{logit}(\pi_{i} ) = \log\left( \frac{\pi_{i}}{1 - \pi_{i}}\right)  = \log\left( \frac{\frac{\exp(X_{i} \beta)}{1 + \exp(X_{i} \beta)}}{ \frac{1}{1 + \exp(X_{i} \beta)} }   \right) = \log(\exp(X_{i} \beta ))  = X_{i} \beta $
   \end{itemize}
\pause
\medskip

  \item Must be monotonic and differentiable
  \item This allows us to express the \alert{mean function} as: $\mu_i = g^{-1}(X_i^\top\beta)$

\end{itemize}

}

\frame{

  \frametitle{Likelihood Function, Score, and Information Matrix}
  \begin{itemize}
  \item Log-likelihood function:
    $$l_n(\theta, \phi; Y) \ = \ \sum_{i=1}^n
    \left\{\frac{Y_i\theta_i-b(\theta_i)}{a(\phi)} +
    c(Y_i,\phi)\right\}$$
\pause
\vspace{-.5\baselineskip}
  \item Recall our notation: $\mu_i = \text{E}[Y_i\mid X_i]$ and $V_i = \text{Var}[Y_i\mid X_i]$
\pause
  \item Score:
{\small
    \begin{equation*}
      s(\beta) \ = \ \frac{\partial l_n(\theta, \phi;Y)}{\partial \beta}
      \ = \ \sum_{i=1}^n \frac{Y_i - b^\prime(\theta_i)}{a(\phi)b^{\prime\prime}(\theta_i)}
         \frac{\partial \mu_i}{\partial \beta}
      \ = \ \sum_{i=1}^n \frac{Y_i - \mu_i}{V_i}\left(\frac{\partial\mu_i}{\partial X_i^\top\beta}\right)X_i
    \end{equation*}
}
\pause
\vspace{-.5\baselineskip}
 \item Information:
{\small
   $$I(\beta) \ = \ -\text{E}\left[H(\beta)\right] \ = \ -\text{E}\left(\frac{\partial^2 l_n(\theta, \phi; Y)}{\partial\beta\partial\beta^\top}\right) \ = \ \sum_{i=1}^n \frac{1}{V_i}\left(\frac{\partial\mu_i}{\partial X_i^\top\beta}\right)^2 X_i X_i^\top$$
}
 % \item Canonical link (i.e., $X_i^\top\beta = \theta_i$): the expected information
 %   $=$ the observed information
\pause
\vspace{-.5\baselineskip}
 \item Exercise: Check these hold for logit, probit, Poisson, etc.!
 \end{itemize}

}


%\frame{
%  \frametitle{Iterated Weighted Least Squares}
%\small
%  \begin{itemize}

%  \item Numerically maximize $l_n(\theta,\phi; Y)$ by Fisher scoring:
%  {\footnotesize
%    \begin{eqnarray*}
%       \beta^{(t+1)}
%      & = & \beta^{(t)} + \Omega(\beta^{(t)})^{-1} s_n(\beta^{(t)})\\
%\pause
%      & = & \beta^{(t)} + \Bigg(\sum_{i=1}^n
%      \underbrace{\frac{1}{V_ig^\prime(\mu_i^{(t)})^2}}_{\equiv \ w_i^{(t)}} X_i X_i^\top\Bigg)^{-1}
%       \sum_{i=1}^n (Y_i-\mu_i^{(t)})\underbrace{\frac{1}{V_ig^\prime(\mu_i^{(t)})^2}}_{\equiv \ w_i^{(t)}}
%       g^\prime(\mu_i^{(t)}) X_i \\
%      & = & \l(\sum_{i=1}^n w_i^{(t)} X_i X_i^\top\r)^{-1} \sum_{i=1}^n w_i^{(t)} X_i
%      \underbrace{\l\{(Y_i-\mu_i^{(t)})g^\prime(\mu_i^{(t)}) + X_i^\top\beta^{(t)}\r\}}_{\equiv \ Z_i^{(t)}} \\
%      & = &  \l(\sum_{i=1}^n w_i^{(t)} X_iX_i^\top \r)^{-1}
%           \l(\sum_{i=1}^n w_i^{(t)} X_i Z_i^{(t)}   \r)
%    \end{eqnarray*}
%  }
%  which resembles the weighted least squares!
%\pause
%  \item This implies any GLM can be estimated by a modified OLS routine \\
%  (a big computational advantage, esp. a few decades ago)
%  \item We can also obtain useful statistics analogous to OLS, e.g. \alert{hat values}
%  \end{itemize}


%}


\begin{frame}
\frametitle{Assessing Goodness of Fit for GLMs}
  \begin{itemize}
    \item Null model: Predict every observation by sample mean $\bar Y$ \\
      $\Rightarrow$ one parameter, maximum data reduction
    \item \alert{Saturated model}: Predict every observation by its own value $Y_i$ \\
      $\Rightarrow$ $n$ parameters, no data reduction
    \item A useful model sits somewhere in between
\medskip
\begin{displaymath}\ShowOutlines
    \xymatrix{ \fbox{\small  $\begin{array}{c} \text{Null Model} \\
                                  (\hat\mu_i = \bar{Y}) \\
                                  K = 1 \end{array}$} \ar@{<-}[r]
                    & \ar@{-}[r]
                    & \ar@{-}[r]^{\text{{\small Useful Models}}}_{1 < K < n}
                    & \ar@{-}[r]
                    & \ar@{->}[r]
                    & \fbox{\small  $\begin{array}{c} \text{Saturated Model} \\
                                         (\hat\mu_i = Y_i) \\
                                         K = n \end{array}$} }
\end{displaymath}
\medskip
\pause
   \item Goodness of fit can be measured by comparing the model likelihood
      to the saturated model likelihood
\end{itemize}
\end{frame}



\begin{frame}
  \frametitle{Analysis of Deviance}
\begin{itemize}
   \item \alert{Scaled Deviance} (Unscaled: $\phi \times $ Deviance $D(Y, \hat{\theta})$):
     \begin{eqnarray*}
     D^\ast(Y; \hat\theta) & \equiv & 2 \{l_n(\tilde\theta;Y,\phi) - l_n(\hat\theta;Y,\phi)\} \\
      & = & 2\sum_{i=1}^n
      \left\{Y_i(\tilde\theta_i - \hat\theta_i) - (b(\tilde\theta_i) - b(\hat\theta_i)) \right\}\big/a(\phi)
      \end{eqnarray*}
{\small
     where $\left\{ \begin{array}{ll}
        \hat\theta_i \ = \ \theta(\hat\mu_i) & \text{(estimate from the model of interest)} \\
      \tilde\theta_i \ = \ \theta(Y_i) & \text{(``estimate'' from the saturated model)}
        \end{array} \right.$
}
\pause
\medskip
    \item Note that $D^\ast$ is a likelihood-ratio statistic
    \item This implies $D^\ast(Y; \hat\theta) \stackrel{approx.}{\sim} \chi^2_{n-k}$
      if the model fits the data well
\pause
    \item We can also compare models: $D^\ast_1 - D^\ast_2 \sim \chi^2_{k_1 - k_2}$ (LR test)
\medskip
\pause
    \item McFadden's \alert{pseudo-$R^2$}:
      $$\tilde R^2 = \frac{l_n(\hat\theta; Y,\phi) - l_n(\theta(\bar Y); Y,\phi)}
                    {l_n(\tilde\theta; Y,\phi) - l_n(\theta(\bar Y); Y,\phi)}
             = 1 - \frac{D^\ast(Y; \hat\theta)}{D^\ast(Y; \theta(\bar Y))} $$
where $\bar\theta_i \ = \ \theta(\bar Y)$ for all $i$
  \end{itemize}
\end{frame}


\begin{frame}
\frametitle{Unscaled Deviance for Normal GLMs}

\begin{eqnarray}
\hat{\theta}_{i} & = & \boldsymbol{X}_{i}^{'} \boldsymbol{\beta} \nonumber \\
b(\hat{\theta}_{i}) & = & \hat{\theta}^2/2 \nonumber \\
\text{Saturated model} & \Rightarrow & \tilde{\theta}_{i} = y_{i} ; b(\tilde{\theta}_{i}) = y_{i}^2/2 \nonumber \\
\text{Deviance} & = & 2 \sum_{i=1}^{N}\left[y_{i}(y_{i} - \hat{\mu}_{i}) - y_{i}^2/2 + \hat{\mu}_{i}^2/2)   \right] = \sum_{i=1}^{N} (y_{i} - \hat{\mu}_{i})^2 \nonumber
\end{eqnarray}




\end{frame}



\begin{frame}
\frametitle{Deviance for Poisson}

Saturated model $\tilde{\lambda}_{i} = y_{i}$.  This implies a log-likelihood of:

\begin{eqnarray}
\tilde{\theta}_{i} & =& \log y_{i} \nonumber \\
b(\tilde{\theta})_{i} & = & \exp(\log y_{i}) = y_{i} \nonumber \\
\text{Deviance} & = & 2 \sum_{i=1}^{N}\left[y_{i} \log \frac{y_{i}}{\hat{\lambda}_{i}} - y_{i} + \hat{\lambda}_{i}  \right] \nonumber
\end{eqnarray}






\end{frame}




\frame{

\frametitle{Residuals and Model Checking}
\small
\vspace{-.25\baselineskip}
\begin{itemize}
\item In normal linear regression, $D = \mathcal{X}^2 = \sum_{i=1}^n \hat\epsilon_i^2 = RSS$
\pause
\item This suggests the following generalization of residuals for GLM:

\begin{enumerate}

\item \alert{Deviance residual}:
  $$ \hat\epsilon_i^D \ \equiv \ \text{sign}(Y_i - \hat\mu_i)\sqrt{d_i}$$
  where $d_i$ is the deviance for the $i$th observation

\medskip

\item \alert{Pearson residual}:
  $$ \hat\epsilon_i^P \ \equiv \ \frac{\sqrt{\omega_i}(Y_i - \hat\mu_i)}{\sqrt{b^{\prime\prime}(\hat\theta_i)}}$$

\end{enumerate}
\medskip

\pause
\item These residuals have approximately the same properties as OLS residuals when $N$ is large
%\item Moreover, numerical maximization of likelihood produces something equivalent to \alert{hat values} as a byproduct

\pause
\item Thus, most \alert{regression diagnostics} for linear models also work for GLMs:
  \begin{itemize}
  \item Plotting standardized and studentized residuals
  \item Analyze influence points and outliers
  \item Added-variable plots, component-residual plots, etc.
  \end{itemize}

\end{itemize}


}



\begin{frame}[fragile]
\frametitle{Example: Democracy and War Involvement}
Benoit (1996):
\begin{itemize}
 \item $Y_i$: \# of involvement in international wars, 1960--80
 \item $X_i$: democracy (Freedom House score), population, military capacity, economic interdependence
\end{itemize}
\pause
\vspace{-1.75\baselineskip}
\begin{columns}
\column{0.25\textwidth}
\begin{center}
\includegraphics[height=.7\textheight]{war-hist.pdf}
\end{center}
\pause
\column{0.375\textwidth}
{\scriptsize
\quad \qquad Coefficients:
\begin{verbatim}
               Est. s.e.
    (Int.)    -3.97 1.62 *
    fh73      -0.06 0.04
    lpopln70   0.62 0.30 *
    lmilwp70   1.29 0.45 **
    ecintdep  -1.28 1.11

               z    p
              -2.45 0.014
              -1.49 0.136
               2.07 0.039
               2.85 0.004
              -1.16 0.247
\end{verbatim}
}
\pause
\column{0.4\textwidth}
\begin{center}
\includegraphics[height=.7\textheight]{war-pred-po.pdf}
\end{center}

\end{columns}

\end{frame}




\begin{frame}
\frametitle{Example: Democracy and War Involvement}
Plotting $\hat\epsilon_i^P$ against $X_{ij}$ and $X_i^\top\hat\beta$:
\begin{center}
\includegraphics[width=\textwidth]{war-residplots.pdf}
\end{center}
\pause
\begin{itemize}
\item Evidence of mild nonlinearity (quadratic) for log population
\item Heavy skew to the right:
  \begin{itemize}
  \item Nonnegativity of the outcome variable $+$ small sample size
  \item Potentially alleviated by a ``zero inflation'' model
  \end{itemize}
\end{itemize}
\end{frame}

\begin{frame}
\frametitle{Example: Democracy and War Involvement}
Studentized residuals and hat values:
\begin{center}
\includegraphics[width=.5\textwidth]{war-rstudentdens.pdf}
\includegraphics[width=.5\textwidth]{war-influence.pdf}
\end{center}
\pause
\begin{itemize}
\item The density of studentized residuals confirms the right skew
\item No obvious outliers (except perhaps Israel)
\end{itemize}

\end{frame}


\begin{frame}
\huge

Model fit


\end{frame}





\end{document}


\item Suppose we have $Q$ restrictions for $p$ covariates.  Include this in matrix $\boldsymbol{R}$.  Under $H_{0}$ we have:

\begin{eqnarray}
H_{0} & : &  \boldsymbol{R} \boldsymbol{\beta}  = r \nonumber \\
H_{1} & : & \boldsymbol{R} \boldsymbol{\beta} \neq r \nonumber
\end{eqnarray}

\item Under the null:
\begin{eqnarray}
\sqrt{n}(\widehat{\boldsymbol{\beta}}_{UR} - \boldsymbol{\beta}) & \rightarrow^{D} & \text{Normal}\left(0, \widehat{\text{Var}(\widehat{\boldsymbol{\beta}}_{UR})}\right) \nonumber \\
R \sqrt{n}(\widehat{\boldsymbol{\beta}}_{UR} - \boldsymbol{\beta}) =  & \rightarrow^{D} & \text{Normal}(0, \boldsymbol{R}\widehat{\text{Var}(\widehat{\boldsymbol{\beta}}_{UR})} \boldsymbol{R}^{'})\nonumber
\end{eqnarray}
\end{itemize}

\end{frame}
\end{document}
